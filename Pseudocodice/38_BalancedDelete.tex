\begin{algorithm}
\DontPrintSemicolon
\SetKwComment{comment}{$\%$}{}
\SetKw{Nil}{nil}
\SetKw{Return}{return}
\SetKw{New}{new}
\SetKw{Int}{int}
\SetKw{And}{and}
\SetKw{Delete}{delete}
\SetKwData{Tree}{Tree}
\SetKwData{Red}{RED}
\SetKwData{Black}{BLACK}
\SetKwData{Item}{Item}
\SetKwFunction{Value}{value}
\SetKwFunction{Key}{key}
\SetKwFunction{Left}{left}
\SetKwFunction{Right}{right}
\SetKwFunction{Color}{color}
\SetKwFunction{BalancedDelete}{balancedDelete}
\SetKwFunction{RotateLeft}{rotateLeft}
\SetKwFunction{RotateRight}{rotateRight}
\caption{\protect\BalancedDelete{\protect\Tree T, \protect\Tree t}}

\While{T $\neq$ t \And t.\Color == \Black}{
	\begin{multicols}{2}
	\Tree p = t.\Parent	\comment{Padre}
	\uIf{t == p.\Left}{
		\Tree f = p.\Right	\comment{Fratello}
		\Tree ns = f.\Left	\comment{Nipote sinistro}
		\Tree nd = f.\Right	\comment{Nipote destro}
		\uIf{f.\Color == \Red}{	\comment{Caso 1a}
			p.\Color = \Red\;
			f.\Color = \Black\;
			\RotateLeft{p}\;
			\comment{t viene lasciato inalterato, pertanto si ricade nei casi 2, 3 o 4}
		}
		\Else{
			\uIf{ns.\Color == nd.\Color == \Black}{	\comment{Caso 2a}
				f.\Color = \Red\;
				t = p\;
			}
			\uElseIf{ns.\Color == \Red \And nd.\Color == \Black}{	\comment{Caso 3a}
				ns.\Color = \Black\;
				f.\Color = \Red\;
				\RotateRight{f}\;
				\comment{t viene lasciato inalterato, pertanto si ricade nel caso 4}
			}
			\ElseIf{nd.\Color == \Red}{	\comment{Caso 4a}
				f.\Color = p.\Color{}\;
				p.\Color = \Black\;
				nd.\Color = \Black\;
				\RotateLeft{p}\;
				t = T\;
			}
		}
	}
	\Else{
		\Tree f = p.\Left	\comment{Fratello}
		\Tree ns = f.\Left	\comment{Nipote sinistro}
		\Tree nd = f.\Right	\comment{Nipote destro}
		\uIf{f.\Color == \Red}{	\comment{Caso 1b}
			p.\Color = \Red\;
			f.\Color = \Black\;
			\RotateRight{p}\;
			\comment{t viene lasciato inalterato, pertanto si ricade nei casi 2, 3 o 4}
		}
		\Else{
			\uIf{ns.\Color == nd.\Color == \Black}{	\comment{Caso 2b}
				f.\Color = \Red\;
				t = p\;
			}
			\uElseIf{nd.\Color == \Red \And ns.\Color == \Black}{	\comment{Caso 3b}
				nd.\Color = \Black\;
				f.\Color = \Red\;
				\RotateLeft{f}\;
				\comment{t viene lasciato inalterato, pertanto si ricade nel caso 4}
			}
			\ElseIf{ns.\Color == \Red}{	\comment{Caso 4a}
				f.\Color = p.\Color{}\;
				p.\Color = \Black\;
				ns.\Color = \Black\;
				\RotateRight{p}\;
				t = T\;
			}
		}
	}
\end{multicols}
}
\end{algorithm}
