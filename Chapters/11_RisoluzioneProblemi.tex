\chapter{Risoluzione problemi}
Dato un problema \`e possibile evidenziare quattro fasi, non necessariamente sequenziali, in modo da trovare una soluzione efficiente:
\begin{itemize}
\item Classificazione del problema.
\item Caratterizzazione della soluzione.
\item Tecnica di progetto.
\item Utilizzo di strutture dati.
\end{itemize}
\section{Classificazione dei problemi}
\subsubsection{Problemi decisionali}
Determinare se un dato di ingresso soddisfa una certa propriet\`a.
\subsubsection{Problemi di ricerca}
\`E presente uno spazio di ricerca, un insieme di soluzioni possibili e una soluzione ammissibile  che rispetta certi vincoli.
\subsubsection{Problemi di ottimizzazione}
Ogni soluzione viene associata ad una funzione di costo, si vuole trovare la soluzione di costo minimo.
\subsubsection{Problemi di approssimazione}
Questo tipo di problemi nasce quando trovare la soluzione ottima \`e computazionalmente impossibile e pertanto ci si accontenta di una soluzione approssimata con costo basso, ma l'ottimalit\`a \`e sconosciuta.
\section{Definizione matematica del problema}
\`E fondamentale definire il problema dal punto id vista matematico, spesso la formulazione \`e banale ma pu\`o suggerire una prima soluzione. 
\section{Tecniche di soluzione problemi}
\subsubsection{Divide-et-impera}
Un problema viene suddiviso in sotto-problemi indipendenti che vengono risolti ricorsivamente (top-down) nell'ambito di problemi di decisione e ricerca.
\subsubsection{Programmazione dinamica}
La soluzione viene costruita bottom-up a partire da un insieme di sotto-problemi potenzialmente ripetuti nell'ambito dei problemi di ottimizzazione.
\subsubsection{Memoization}
Una versione top-down della programmazione dinamica.
\subsubsection{Tecnica greedy}
Approccio ingordo: si fa sempre la scelta localmente ottima.
\subsubsection{Backtrack}
Si procede per tentativi tornando quando necessario all'indietro.
\subsubsection{Ricerca locale}
La soluzione ottima viene trovata migliorando via via soluzioni esistenti.
\subsubsection{Algoritmi probabilistici}
Algoritmi in cui certe scelte avvengono in maniera casuale. 
