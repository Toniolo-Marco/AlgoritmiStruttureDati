\chapter{Algoritmi greedy}
Gli algoritmi greedy vengono utilizzati per la risoluzione di problemi di ottimizzazione che eseguono una sequenza di decisioni. Se la programmazione dinamica valuta in maniera bottom-up
valutando tutte le decisioni possibili evitando di ripetere sotto-problemi o decisioni gi\`a percorse gli algoritmi greedy selezionano una sola delle possibili decisioni che sembra
ottima, ovvero quella localmente ottima. In questo caso si rende necessario dimostrare che si ottiene un ottimo globale. La tecnica greedy va applicata se \`e possibile dimostrare che 
esiste una scelta ingorda, ovvero fra le molte scelte possibili ne pu\`o essere facilmente individuata una che porta sicuramente alla soluzione ottima e se il problema ha sottostruttura
ottima, ovvero se fatta tale scelta resta un sottoproblema con la stessa struttura del problema principale. Non tutti i problemi hanno una scelta greedy e in alcuni casi soluzioni ottime
possono comunque essere interessanti.
\section{Insieme indipendente massimale di intervalli}
\paragraph{Input}
Sia $S=\{1, \dots, n\}$ un insieme di intervalli della retta reale. Ogni intervallo $[a_i, b_i[$, con $i\in S$ \`e chiuso a sinistra e aperto a destra. Si indica con $a_i$ il tempo di
inizio e con $b_i$ il tempo di fine. 
\paragraph{Output}
Si vuole trovare un insieme dipendente massimale, ovvero un sottoinsieme di massima cardinalit\`a formato da intervalli tutti disgiunti tra di loro (naturale confronto con l'insieme
indipendente di intervalli pesati).
\subsection{Affrontare il problema}
\subsubsection{Programmazione dinamica}
Si individui una sottostruttura ottima e si scriva una definizione ricorsiva per la dimensione della soluzione ottima con la corrispettiva versione iterativa bottom-up.
\subsubsection{Tecnica greedy}
Si cerchi una possibile scelta ingorda, si dimostri che porta alla soluzione ottima e si scriva un algoritmo ricorsivo o iterativo che effettua sempre la scelta ingorda.
\subsection{Sottostruttura ottima}
Si assuma che gli intervalli siano ordinati per tempo di fine: $b_1\le \cdots\le b_n$ e si definisca il sottoproblema $S[i\dots j]$ come l'insieme di intervalli che iniziano dopo la fine
di $i$ e finiscono prima dell'inizio di $j$: $S[i\dots j] = \{k|b_j\le a_k < b_k\le a_j\}$. Si aggiungano due intervalli fittizi: lo zeresimo $b_0=-\infty$ e l'$n+1$-esimo 
$a_{n+1} = +\infty$, Il problema iniziale corrisponde a $S[0, n+1]$. 
\subsubsection{Teorema}
\paragraph{Enunciato}
Si supponga che $A[i\dots j]$ sia una soluzione ottimale di $S[i\dots j]$ e sia $k$ un intervallo che appartiene a $A[i\dots j]$, allora:
\begin{itemize}
	\item Il problema $S[i\dots j]$ viene diviso in due sottoproblemi:
		\begin{itemize}
			\item $S[i\dots k]$, gli intervalli che finiscono prima di $k$.
			\item $S[k\dots j]$, gli intervalli che iniziano dopo $k$.
		\end{itemize}
	\item $A[i\dots j]$ contiene le soluzioni ottimali di $S[i\dots k]$ e $S[k\dots j]$.
		\begin{itemize}
			\item $A[i\dots j]\cap S[i\dots k]$ \`e la soluzione ottimale di $S[i\dots k]$.
			\item $A[i\dots j]\cap S[k\dots j]$ \`e la soluzione ottimale di $S[k\dots j]$.
		\end{itemize}
\end{itemize}
\paragraph{Dimostrazione}
Si dimostra utilizzando il metodo cut-and-paste.
\subsection{Definizione ricorsiva del costo della soluzione}
\subsubsection{Definizione ricorsiva della soluzione}
$$A[i\dots j] = A[i\dots k] \cup \{k\}\cup A[k\dots j]$$
\subsubsection{Definizione ricorsiva del costo}
$k$ viene determinato analizzando tutte le possibilit\`a. Sia pertanto $DP[i][j]$ la dimensione del pi\`u grande sottoinsieme $A[i\dots j]\subseteq S[i\dots j]$ di intervalli regolari:
$$ DP[i][j] = \begin{cases}
	0 & S[i\dots j] = \emptyset\\
	\max\limits_{k\in S[i\dots j]}\{DP[i][k]+DP[k][j]+1\} & altrimenti\\
\end{cases}$$
Questa soluzione permette di scrivere un algoritmo basato su programmazione dinamica con complessit\`a $O(n^3)$: bisogna risolvere tutti i problemi con $i<j$, con costo $O(n)$ per 
sottoproblema nel caso peggiore.
\subsection{Scelta ingorda}
L'algoritmo risulta migliorabile in quanto non \`e necessario analizzare tutti valori possibili di $k$.
\subsubsection{Teorema}
\paragraph{Enunciato}
Sia $S[i\dots j]$ un sottoproblema non vuoto e $m$ l'intervallo di $S[i\dots j]$ con il minor tempo di fine, allora il sottoproblema $S[i\dots m]$ \`e vuoto e $m$ \`e compreso in 
qualche soluzione ottima di $S[i\dots j]$.
\paragraph{Dimostrazione}
Si sa per definizione di intervallo che $a_m<b_m$ e che essendo che $m$ ha minor tempo di fine $\forall k\in S[i\dots j]:b_m\le b_k$. Ne consegue che $\forall k\in S[i\dots k]:a_m<b_k$
per la propriet\`a transitiva. Se nessun intervallo in $S[i\dots j]$ termina prima di $a_m$ allora $S[i\dots m] = \emptyset$. Sia ora $A'[i\dots j]$ una soluzione ottima di $S[i\dots j]$
e $m'$ l'intervallo con minor tempo di fine in $A'[i\dots j]$. Sia ora $A[i\dots j] = A-[j\dots j]-\{m'\}\cup \{m\}$ una nuova soluzione ottenuta togliendo $m'$ e aggiungendo $m$ a
$A'[i\dots j]$. $A[i\dots j]$ \`e una soluzione ottima che contiene $m$ in quanto ha la stessa dimensione di $A'[i\dots j]$ e gli intervalli sono indipendenti. 
\subsubsection{Conseguenze}
Con questo teorema non \`e pi\`u necessario analizzare tutti i valori possibili di $k$: faccio una scelta ingorda ma sicura selezionando l'attivit\`a $m$ con il minor tempo di fine.
In questo modo non \`e pi\`u necessario analizzare due sottoproblemi eliminando tutte le attivit\`a non compatibili con la scelta ingorda, rimane da risolvere solo $S[m\dots j]$. 
\subsubsection{Implementazione}
\begin{algorithm}[H]
\DontPrintSemicolon
\SetKwComment{comment}{$\%$}{}

\SetKw{Int}{int}
\SetKw{Boolean}{boolean}
\SetKw{New}{new}
\SetKw{True}{true}
\SetKw{False}{false}
\SetKw{Not}{not}
\SetKw{And}{and}
\SetKw{Or}{or}
\SetKw{Down}{down}
\SetKw{To}{to}
\SetKw{New}{new}
\SetKw{Return}{return}
\SetKw{Nil}{nil}
\SetKw{Print}{print}
\SetKw{Void}{void}

\SetKwData{Item}{Item}
\SetKwData{Mfset}{Mfset}
\SetKwData{Graph}{Graph}
\SetKwData{N}{n}
\SetKwData{Space}{ }
\SetKwData{Parent}{parent}
\SetKwData{Rank}{rank}
\SetKwData{Set}{Set}
\SetKwData{List}{List}

\SetKwFunction{Max}{max}
\SetKwFunction{Insert}{insert}
\SetKwFunction{ListCos}{List}
\SetKwFunction{Head}{head}
\SetKwFunction{Len}{len}
\SetKwFunction{SetCos}{Set}


\SetKwProg{Fn}{}{}

\SetKwFunction{IndipendentSet}{indipendentSet}
\SetKwFunction{}{}
\SetKwFunction{}{}
\SetKwFunction{}{}
\SetKwFunction{}{}
\SetKwFunction{}{}

\SetKw{Matrix}{matrix}

\caption{\protect\Set \protect\IndipendentSet{\protect\Int[] a, \protect\Int[] b}}
$\{$ordina $a$ e $b$ in modo che $b[1]\le b[2]\le\cdots\le b[n]$ $\}$\;
\Set S = \SetCos{}\;
S.\Insert{1}\;
\Int last = 1\;
\For{\Int i = 2 \To n}{
	\If{a[i] $\ge$ b[last]}{
		S.\Insert{i}\;
		last = i\;
	}
}
\Return S\;
\end{algorithm}

La complessit\`a \`e $O(n\log n)$ se l'input non \`e ordinato, $O(n)$ se ordinato. 
\section{Approccio a partire da programmazione dinamica}
Si cerca di risolvere il problema della selezione delle attivit\`a tramite programmazione dinamica individuando una sottostruttura ottima e scrivendo una definizione ricorsiva per la
dimensione della soluzione ottima. Si dimostra la propriet\`a della scelta greedy: per ogni sottoproblema esiste una soluzione ottima che contiene la scelta greedy e si scrive un
algoritmo iterativo che effettua sempre la scelta ingorda.
\section{Problema del resto}
\paragraph{Input}
Un insieme di tagli di monete memorizzati in un vettore di interi positivi $t[1\dots n[$ e un intero $R$ rappresentante il resto che si deve restituire.
\paragraph{Definizione del problema}
Si deve trovare il pi\`u piccolo numero intero di pezzi necessari per dare un resto di $R$ centesimi utilizzando i tagli dati assumendo di avere un numero illimitato di monete per ogni 
taglio, ovvero trovare un vettore $x$ di interi non negativi tale che $R=\sum\limits_{i = 1}^nx[i]\cdot t[i]$ e $m=\sum\limits_{i = 1}^nx[i]$ ha un valore minimo.
\subsection{Soluzione basata su programmazione dinamica}
\subsubsection{Sottostruttura ottima}
Sia $S(i)$ il problema per dare un resto pari a $i$ e $A(i)$ la soluzione ottima del problema $S(i)$ rappresentata da un multi-insieme e $j\in A(1)$, allora $S(i-t[j])$ \`e un 
sottoproblema di $S(i)$ la cui soluzione ottima \`e data da $A(i)-\{j\}$. 
\paragraph{Definizione ricorsiva}
Sia la tabella di programmazione dinamica $DP[0\dots R]$ e $DP[i]$ il minimo numero di monete per risolvere il problema $S(i)$.
$$DP[i] = \begin{cases}
	0 & i = 0 \\
	\min\limits_{1\le j\le n}\{DP[i-t[j] | t[j]\le i\} + 1 & i>0\\
\end{cases}$$
\subsection{Implementazione}
\begin{algorithm}[H]
\DontPrintSemicolon
\SetKwComment{comment}{$\%$}{}

\SetKw{Int}{int}
\SetKw{Boolean}{boolean}
\SetKw{New}{new}
\SetKw{True}{true}
\SetKw{False}{false}
\SetKw{Not}{not}
\SetKw{And}{and}
\SetKw{Or}{or}
\SetKw{Down}{down}
\SetKw{To}{to}
\SetKw{New}{new}
\SetKw{Return}{return}
\SetKw{Nil}{nil}
\SetKw{Print}{print}
\SetKw{Void}{void}

\SetKwData{Item}{Item}
\SetKwData{Mfset}{Mfset}
\SetKwData{Graph}{Graph}
\SetKwData{N}{n}
\SetKwData{Space}{ }
\SetKwData{Parent}{parent}
\SetKwData{Rank}{rank}
\SetKwData{Set}{Set}
\SetKwData{List}{List}

\SetKwFunction{Max}{max}
\SetKwFunction{Insert}{insert}
\SetKwFunction{ListCos}{List}
\SetKwFunction{Head}{head}
\SetKwFunction{Len}{len}
\SetKwFunction{SetCos}{Set}


\SetKwProg{Fn}{}{}

\SetKwFunction{Resto}{resto}
\SetKwFunction{}{}
\SetKwFunction{}{}
\SetKwFunction{}{}
\SetKwFunction{}{}
\SetKwFunction{}{}

\SetKw{Matrix}{matrix}

\caption{\protect\Int[] \protect\Resto{\protect\Int[] t, \protect\Int n, \protect\Int R}}
\Int[] DP = \New \Int$[0\dots R]$\;
\Int[] S = \New \Int$[0\dots R]$\;
DP[0] = 0\;
\For{\Int i = 1 \To R}{
	DP[i] = +$\infty$\;
	\For{\Int j = 1 \To n}{
		\If{i $\ge$ t[j] \And DP[i - t[j]] + 1 $<$ DP[i]}{
			DP[i] = DP[i - t[j]] + 1\;
			S[i] = j\;
		}
	}
}
\Int[] x = \New \Int$[1\dots n]$\;
\For{\Int i = 1 \To n}{
	x[i] = 0\;
}
\While{R $>$ 0}{
	x[S[r]] = x[S[r]] + 1\;
	R -= t[S[R]]\;
}
\Return x\;

\end{algorithm}

Questo algoritmo ha complessit\`a $O(nR)$. 
\subsection{Scelta greedy}
\`E possibile pensare ad una soluzione greedy selezionando la moneta $j$ pi\`u grande tale per cui $t[j]\le R$ e risolvere il problema $S(R - t[j])$. 
\subsubsection{Implementazione}
\begin{algorithm}[H]
\DontPrintSemicolon
\SetKwComment{comment}{$\%$}{}

\SetKw{Int}{int}
\SetKw{Boolean}{boolean}
\SetKw{New}{new}
\SetKw{True}{true}
\SetKw{False}{false}
\SetKw{Not}{not}
\SetKw{And}{and}
\SetKw{Or}{or}
\SetKw{Down}{down}
\SetKw{To}{to}
\SetKw{New}{new}
\SetKw{Return}{return}
\SetKw{Nil}{nil}
\SetKw{Print}{print}
\SetKw{Void}{void}

\SetKwData{Item}{Item}
\SetKwData{Mfset}{Mfset}
\SetKwData{Graph}{Graph}
\SetKwData{N}{n}
\SetKwData{Space}{ }
\SetKwData{Parent}{parent}
\SetKwData{Rank}{rank}
\SetKwData{Set}{Set}
\SetKwData{List}{List}

\SetKwFunction{Max}{max}
\SetKwFunction{Insert}{insert}
\SetKwFunction{ListCos}{List}
\SetKwFunction{Head}{head}
\SetKwFunction{Len}{len}
\SetKwFunction{SetCos}{Set}


\SetKwProg{Fn}{}{}

\SetKwFunction{Resto}{resto}
\SetKwFunction{}{}
\SetKwFunction{}{}
\SetKwFunction{}{}
\SetKwFunction{}{}
\SetKwFunction{}{}

\SetKw{Matrix}{matrix}

\caption{\protect\Int[] \protect\Resto{\protect\Int[] t, \protect\Int n, \protect\Int R}}
\Int[] x = \New \Int$[1\dots n]$\;
$\{$ordina le monete in modo decrescente$\}$\;
\For{\Int i = 1 \To n}{
	x[i] = $\lfloor\frac{R}{t[i]}\rfloor$\;
	R -= x[i] $\cdot$ t[i]\;
}
\Return x\;
\end{algorithm}

Questo algoritmo cha complessit\`a $O(n\log n)$ se l'input non \`e ordinato e $O(n)$ se \`e ordinato.
\section{Approccio greedy senza programmazione dinamica}
Si evidenzino i passi di decisione trasformando il problema di ottimizzazione in un problema di scelte successive, scegliendo una possibile scelta ingorda dimostrando che rispetta il
``principio della scelta ingorda". Si evidenzia la sottostruttura ottima dimostrando che la soluzione ottima del problema rimanente dopo la scelta ingorda pu\`o essere unito a tale
scelta. La struttura del codice \`e top-down, anche iterativamente. Pu\`o essere necessario pre-processare l'input. 
\section{Scheduling}
\paragraph{Input}
Si supponga di avere un processore e $n$ job da eseguire su di esso, ognuno caratterizzato da un tempo di esecuzione $t[i]$ eseguito a priori.
\paragraph{Problema}
Si deve trovare una sequenza di esecuzione che minimizzi il tempo di completamento medio. Dato un vettore $A[1\dots n]$ contenente una permutazione di $\{1\dots n\}$ il tempo di 
completamento dell'$h$-esimo job nella permutazione \`e $T_A(h) = \sum\limits_{i = 1}^h t[A[i]]$. 
\subsection{Dimostrazione di correttezza}
\subsubsection{Teorema di scelta greedy}
\paragraph{Enunciato}
Esiste una soluzione ottima $A$ in cui il job con minore tempo di fine $m$ si trova in prima posizione: $A[1] = m$.
\paragraph{Dimostrazione}
Si consideri la permutazione ottima $B = [B[1], \dots, B[m-1], B[m], B[m+1], \dots, B[n]]$ e sia $m$ la posizione in cui si trova in $B$ il job con il minor tempo di fine. Si consideri
una permutazione in cui i job in posizione $1$ vengono scambiati: $A =[B[m], B[2] \dots, B[m-1], B[1], B[m+1], \dots, B[n]]$. Il tempo di completamento medio di $A$ \`e minore o uguale
al tempo di completamento medio di $B$. In quanto i job in posizione $1, \dots, m-1$ in $A$ hanno tempo di completamento $\le$ dei job in posizione $1, \dots, m-1$ in $B$ e i job in
posizione $m, \dots, n$ in $A$ hanno tempo di completamento uguale ai job in posizione $m, \dots, n$ in $B$. Essendo $B$ ottima, $A$ non pu\`o avere tempo di completamento medio minore
e pertanto anche $A$ \`e ottima. 
\subsubsection{Teorema della sottostruttura ottima}
\paragraph{Enunciato}
Sia $A$ una soluzione ottima di un problema con $n$ job in cui il job con minor tempo di fine $m$ si trova in prima posizione. La permutazione dei seguenti $n-1$ job in $A$ \`e una
soluzione ottima al sottoproblema in cui il job $m$ non viene considerato. 
\section{Problema dello zaino reale}
\paragraph{Input}
Un intero positivo $C$, la capacit\`a dello zaino e $n$ oggetti, tali che l'oggetto $i$-esimo \`e caratterizzato da un profitto $p_i\in\mathbb{Z}^+$ e un peso $w_i\in\mathbb{Z}^+$.
\paragraph{Problema}
Trovare un sottoinsieme $S$ di $\{1,\dots n\}$ di oggetti tale che il loro peso totale non superi la capacit\`a massima e il loro profitto totale sia massimo. Nella variante dello zaino
reale \`e possibile prendere frazioni di oggetti.
\subsection{Approcci greedy possibili}
Gli oggetti si possono ordinare per profitto decrescente, per peso crescente o per profitto specifico ($\frac{p_i}{w_i}$) decrescente. Il terzo approccio funziona unicamente nella
variante dello zaino reale.
\subsubsection{Implementazione}
\begin{algorithm}[H]
\DontPrintSemicolon
\SetKwComment{comment}{$\%$}{}

\SetKw{Int}{int}
\SetKw{Float}{float}
\SetKw{Boolean}{boolean}
\SetKw{New}{new}
\SetKw{True}{true}
\SetKw{False}{false}
\SetKw{Not}{not}
\SetKw{And}{and}
\SetKw{Or}{or}
\SetKw{Down}{down}
\SetKw{To}{to}
\SetKw{New}{new}
\SetKw{Return}{return}
\SetKw{Nil}{nil}
\SetKw{Print}{print}
\SetKw{Void}{void}

\SetKwData{Item}{Item}
\SetKwData{Mfset}{Mfset}
\SetKwData{Graph}{Graph}
\SetKwData{N}{n}
\SetKwData{Space}{ }
\SetKwData{Parent}{parent}
\SetKwData{Rank}{rank}
\SetKwData{Set}{Set}
\SetKwData{List}{List}

\SetKwFunction{Max}{max}
\SetKwFunction{Min}{min}
\SetKwFunction{Insert}{insert}
\SetKwFunction{ListCos}{List}
\SetKwFunction{Head}{head}
\SetKwFunction{Len}{len}
\SetKwFunction{SetCos}{Set}


\SetKwProg{Fn}{}{}

\SetKwFunction{Zaino}{zaino}
\SetKwFunction{}{}
\SetKwFunction{}{}
\SetKwFunction{}{}
\SetKwFunction{}{}
\SetKwFunction{}{}

\SetKw{Matrix}{matrix}

\caption{\protect\Float[] \protect\Zaino{\protect\Float[] p, \protect\Float[] w, \protect\Float C,\protect\Float n}}
\Float[] x = \New \Float$[1\dots n]$\;
$\{$ordina $p$ e $w$ in modo che $\frac{p[1]}{w[1]}\ge\cdots\ge\frac{p[n]}{w[n]}$ $\}$\;
\For{\Int i = 1 \To n}{
	x[i] = \Min{$\frac{C}{w[i]}$, 1}\;
	C -= x[i] $\cdot$ w[i]\;
}
\Return x\;
\end{algorithm}

L'algoritmo ha complessit\`a $O(n\log n)$ se l'input non \`e ordinato, $O(n)$ se \`e ordinato. $x[i]\in[0, 1]$ rappresenta la proporzione dell'oggetto $i$-esimo che deve essere 
prelevata.
\subsection{Correttezza}
Si assuma che gli oggetti siano ordinati per profitto specifico decrescente e $x$ una soluzione ottima. Si supponga che $x[1] < \min(\frac{C}{w[i]}, 1)$ e la proporzione di uno o pi\`u
oggetti \`e ridotta di conseguenza. Si ottiene cos\`i una soluzione di profitto uguale o superiore visto che il profitto specifico dell'oggetto $1$ \`e massimo. 
\section{Problema della compressione}
La compressione si occupa di rappresentare i dati in modo efficiente, impiegando il minore numero di bit per la rappresentazione risparmiando spazio su disco e tempo di trasferimento.
Una possibile tecnica di compressione \`e la codifica di caratteri tramite una funzione di codifica $f:f(c) = x$, dove $c$ \`e un carattere preso da un alfabeto $\Sigma$ e $x$ \`e una
rappresentazione binaria. 
\subsection{Codifica a prefissi}
In un codice a prefisso o senza prefissi nessun codice \`e prefisso di un altro codice, condizione necessaria per la codifica. 
\subsubsection{Rappresentazione ad albero per la decodifica}
L'alfabeto viene rappresentato come un albero binario le cui foglie sono i caratteri dell'alfabeto e tutti gli altri nodi hanno due figli. In base al bit letto si sceglie dove spostarsi
nell'albero fino a trovare il carattere necessario.\\
\begin{algorithm}[H]
\DontPrintSemicolon
\SetKwComment{comment}{$\%$}{}

\SetKw{Int}{int}
\SetKw{Float}{float}
\SetKw{Boolean}{boolean}
\SetKw{New}{new}
\SetKw{True}{true}
\SetKw{False}{false}
\SetKw{Not}{not}
\SetKw{And}{and}
\SetKw{Or}{or}
\SetKw{Down}{down}
\SetKw{To}{to}
\SetKw{New}{new}
\SetKw{Return}{return}
\SetKw{Nil}{nil}
\SetKw{Print}{print}
\SetKw{Void}{void}

\SetKwData{Item}{Item}
\SetKwData{Mfset}{Mfset}
\SetKwData{Graph}{Graph}
\SetKwData{N}{n}
\SetKwData{Space}{ }
\SetKwData{Parent}{parent}
\SetKwData{Rank}{rank}
\SetKwData{Set}{Set}
\SetKwData{List}{List}

\SetKwFunction{Max}{max}
\SetKwFunction{Min}{min}
\SetKwFunction{Insert}{insert}
\SetKwFunction{ListCos}{List}
\SetKwFunction{Head}{head}
\SetKwFunction{Len}{len}
\SetKwFunction{SetCos}{Set}


\SetKwProg{Fn}{}{}

\SetKwFunction{Zaino}{zaino}
\SetKwFunction{}{}
\SetKwFunction{}{}
\SetKwFunction{}{}
\SetKwFunction{}{}
\SetKwFunction{}{}

\SetKw{Matrix}{matrix}

\caption{Algoritmo di decodifica}
si parte dalla radice\;
\While{file non finito}{
	si legge un bit\;
	\uIf{bit $=$ zero}{
		vai nel figlio sinistro\;
	}
	\Else{
		vai a destra\;
	}
	\If{nodo foglia}{
		stampa il carattere\;
		torna alla radice\;
	}
}
\end{algorithm}

\subsection{Definizione formale del problema}
\paragraph{Input}
Un file $F$ composto da caratteri nell'alfabeto $\Sigma$. Sia $T$ un albero che rappresenta la codifica, per ogni $c\in\Sigma$, sia $d_T(c)$ la profondit\`a della foglia che rappresenta
$c$, il codice richieder\`a $d_T(c)$ bit per $c$. Se $f[c]$ \`e il numero di occorrenze di $c$ in $F$, allora la dimensione della codifica \`e $C[F, T] = \sum\limits_{c\in\Sigma}f[c]
\cdot d_T(c)$.
\subsection{Algoritmo di Huffman}
Il principio del codice di Huffman consiste del minimizzare la lunghezza dei caratteri che compaiono pi\`u frequentemente assegnando ai caratteri con la frequenza minore i codici 
corrispondenti ai percorsi pi\`u lunghi all'interno dell'albero. Ogni codice \`e progettato per un file specifico: si ottiene la frequenza di tutti i caratteri, si costruisce il 
codice, si rappresenta il file tramite il codice e si aggiunge al file una rappresentazione del codice per la decodifica. 
\subsubsection{Funzionamento algoritmo}
\begin{itemize}
	\item Si costruisce una lista ordinata di nodi foglia per ogni carattere etichettato con la propria frequenza.
	\item Si rimuovono i due nodi con frequenze minori $f_x$ e $f_y$ creando un nodo padre con etichetta ``-" e frequenza $f_x+f_y$ collegando i due nodi rimossi con il nuovo nodo e
		aggiungendolo alla lista mantenendo l'ordine.
	\item Iterare il secondo punto fino a che rimane un unico nodo nella lista e al termine si etichettano gli archi all'interno dell'albero con bit $0$, $1$.
\end{itemize}
\begin{algorithm}[H]
\DontPrintSemicolon
\SetKwComment{comment}{$\%$}{}

\SetKw{Int}{int}
\SetKw{Float}{float}
\SetKw{Boolean}{boolean}
\SetKw{New}{new}
\SetKw{True}{true}
\SetKw{False}{false}
\SetKw{Not}{not}
\SetKw{And}{and}
\SetKw{Or}{or}
\SetKw{Down}{down}
\SetKw{To}{to}
\SetKw{New}{new}
\SetKw{Return}{return}
\SetKw{Nil}{nil}
\SetKw{Print}{print}
\SetKw{Void}{void}

\SetKwData{Item}{Item}
\SetKwData{Tree}{Tree}
\SetKwData{PriorityQueue}{PriorityQueue}
\SetKwData{Mfset}{Mfset}
\SetKwData{Graph}{Graph}
\SetKwData{N}{n}
\SetKwData{Space}{ }
\SetKwData{Parent}{parent}
\SetKwData{Rank}{rank}
\SetKwData{Set}{Set}
\SetKwData{List}{List}
\SetKwData{F}{f}
\SetKwData{Left}{left}
\SetKwData{Right}{right}

\SetKwFunction{Max}{max}
\SetKwFunction{Min}{min}
\SetKwFunction{Insert}{insert}
\SetKwFunction{ListCos}{List}
\SetKwFunction{Head}{head}
\SetKwFunction{Len}{len}
\SetKwFunction{SetCos}{Set}
\SetKwFunction{TreeCos}{Tree}
\SetKwFunction{MinPriorityQueue}{MinPriorityQueue}
\SetKwFunction{DeleteMin}{deleteMin}


\SetKwProg{Fn}{}{}

\SetKwFunction{Huffman}{huffman}
\SetKwFunction{}{}
\SetKwFunction{}{}
\SetKwFunction{}{}
\SetKwFunction{}{}
\SetKwFunction{}{}

\SetKw{Matrix}{matrix}

\caption{\protect\Tree \protect\Huffman{\protect\Int[] c, \protect\Int[] f, \protect\Int n}}
\PriorityQueue Q = \MinPriorityQueue{}\;
\For{\Int i = 1 \To n}{
	Q.\Insert{f[i], \TreeCos{F[i], c[i]}}\;
}
\For{\Int i = 1 \To n - 1}{
	\Tree $z_1$ = Q.\DeleteMin{}\;
	\Tree $z_2$ = Q.\DeleteMin{}\;
	\Tree $z$ = \TreeCos{$z_1$.\F + $z_2$.\F, \Nil}\;
	$z$.\Left = $z_1$\;
	$z$.\Right = $z_2$\;
	Q.\Insert{$z$.\F. $z$}\;
}
\Return Q.\DeleteMin{}\;
\end{algorithm}

Questo algoritmo ha complessit\`a $O(n\log n)$. 
\subsubsection{Correttezza}
L'output dell'algoritmo di Huffman per un dato file \`e un codice a prefisso ottimo. La propriet\`a della scelta greedy \`e che scegliere i due elementi con la frequenza pi\`u bassa 
conduce sempre ad una soluzione ottimale. La sottostruttura ottima \`e che dato un problema sull'alfabeto $\Sigma$ \`e possibile costruire un sottoproblema con alfabeto pi\`u piccolo. 
\subsubsection{Scelta greedy}
\paragraph{Enunciato} 
Siano $\Sigma$ un alfabeto, $f$ un vettore di frequenze e $x$ e $y$ i due caratteri con frequenza pi\`u bassa. Esiste un codice prefisso ottimo per $\Sigma$ in cui $x$ e $y$ hanno la
stessa profondit\`a massima e i loro codici differiscono solo per l'ultimo bit. 
\paragraph{Dimostrazione}
Si basa sulla trasformazione di una soluzione ottima. Si supponga che esista un codice ottimo $T$ in cui due catatteri $a$ e $b$ con profondit\`a massima siano diversi da $x$ e $y$. 
Si pu\`o assumere senza perdere di generalit\`a che $f[x]\le f[y]\land f[a]\le f[b]$. Essendo le frequenze di $x$ e $y$ minime $f[x]\le f[a]\land f[y]\le f[b]$. Scambiando $x$ con $a$
si ottiene $T'$ e scambiando $y$ con $b$ si ottiene $T''$. Si dimostri ora che $C(f, T'')\le C(f, T')\le C(f, T)$.
\begin{align*}
	C(T) - C(T') & = \sum\limits_{c\in \Sigma}f[c]d_T(c) - \sum\limits_{c\in\Sigma}f[c]d_{T'}(c)\\
	             & = (f[x]d_T(x)+f[a]d_T(a)) - (f[x]d_{T'}(x) + f[a]d_{T'}(a))\\
		     & = (f[x]d_T(x)+f[a]d_T(a)) - (f[x]d_{T}(x) + f[a]d_{T}(a))\\
		     & = (f[a] - f[x])(d_T(a) - d_T(x))\\
		     & \ge 0\\
	C(T') - C(T'') & \ge 0\quad\quad \text{Come sopra}
\end{align*}
Essendo $T$ ottimo allora $C(f, T)\le C(f, T'')$, pertanto anche $T''$ \`e ottimo.
\section{Albero di copertura di peso minimo}
Dato un grafo pesato determinare come interconnettere tutti i suoi nodi minimizzando il costo del peso associato ai suoi archi, ovvero albero di copertura, di connessione di peso minimo
o minimum spanning tree.
\subsection{Definizione del problema}
Sia $G=(V, E)$ un grafo non orientato e connesso e $w:V\times V\rightarrow\mathbb{R}$ una funzione di peso costo di connessione tale che se $[u, v]\in E$ allora $w(u, v)$ \`e il peso 
dell'arco $[u, v]$, se $[u, v]\not\in E$, allora $w(u, v)= + \infty$. Essendo $G$ non orientato $w(u, v) = w(v, u)$.
\subsubsection{Albero di copertura}
Dato un grafo non orientato e connesso un albero di copertura $G$ \`e un sottografo $T=(V, E_T)$ tale che $T$ \`e un albero, $E_T\subseteq E$ e $T$ contiene tutti i vertici di $G$. 
\subsubsection{Output}
Trovare l'albero di copertura il cui peso totale sia minimo rispetto a ogni altro albero di copertura, dove $w(T)=\sum\limits_{[u, v]\in E_T} w(u, v)$. Non \`e detto che sia univoco.
\subsection{Algoritmo generico}
Ci si approccia al problema tentando di accrescere un sottoinsieme $A$ di archi in modo tale che venga sempre rispettato il fatto \`e $A$ \`e un sottoinsieme di qualche albero di 
connessione minimo.
\subsubsection{Arco sicuro}
Un arco $[u, v]$ si dice sicuro per $A$ se $A\cup\{[u, v]\}$ \`e ancora un sottoinsieme di qualche albero di connessione minimo.\\
\begin{algorithm}[H]
\DontPrintSemicolon
\SetKwComment{comment}{$\%$}{}

\SetKw{Int}{int}
\SetKw{Float}{float}
\SetKw{Boolean}{boolean}
\SetKw{New}{new}
\SetKw{True}{true}
\SetKw{False}{false}
\SetKw{Not}{not}
\SetKw{And}{and}
\SetKw{Or}{or}
\SetKw{Down}{down}
\SetKw{To}{to}
\SetKw{New}{new}
\SetKw{Return}{return}
\SetKw{Nil}{nil}
\SetKw{Print}{print}
\SetKw{Void}{void}

\SetKwData{Item}{Item}
\SetKwData{Tree}{Tree}
\SetKwData{PriorityQueue}{PriorityQueue}
\SetKwData{Mfset}{Mfset}
\SetKwData{Graph}{Graph}
\SetKwData{N}{n}
\SetKwData{Space}{ }
\SetKwData{Parent}{parent}
\SetKwData{Rank}{rank}
\SetKwData{Set}{Set}
\SetKwData{List}{List}
\SetKwData{F}{f}
\SetKwData{Left}{left}
\SetKwData{Right}{right}

\SetKwFunction{Max}{max}
\SetKwFunction{Min}{min}
\SetKwFunction{Insert}{insert}
\SetKwFunction{ListCos}{List}
\SetKwFunction{Head}{head}
\SetKwFunction{Len}{len}
\SetKwFunction{SetCos}{Set}
\SetKwFunction{TreeCos}{Tree}
\SetKwFunction{MinPriorityQueue}{MinPriorityQueue}
\SetKwFunction{DeleteMin}{deleteMin}


\SetKwProg{Fn}{}{}

\SetKwFunction{MSTGenerico}{mst-generico}
\SetKwFunction{}{}
\SetKwFunction{}{}
\SetKwFunction{}{}
\SetKwFunction{}{}
\SetKwFunction{}{}

\SetKw{Matrix}{matrix}
\caption{\protect\Set \protect\MSTGenerico{\protect\Graph G, \protect\Int[] w}}
\Set A = $\emptyset$\;
\While{A \Not albero di copertura}{
	trova un arco sicuro $[u, v]$\;
	A = A $\cup\ \{[u, v]\}$\;
}
\Return A\;
\end{algorithm}

\paragraph{Definizioni}
\begin{itemize}
	\item Un taglio $(S, V - S)$ di un grafo non orientato \`e una partizione di $V$ in due sottoinsiemi disgiunti.
	\item Un arco $[u, v]$ attraversa il taglio se $u\in S$ e $v\in V-S$.
	\item Un taglio rispetta un insieme di archi $A$ se nessun arco di $A$ attraversa il taglio.
	\item Un arco che attraversa un taglio \`e leggero nel taglio se il suo peso \`e minimo fra i pesi degli archi che attraversano un taglio.
\end{itemize}
\paragraph{Enunciato}
Sia $G=(V, E)$ un grafo non orientato e connesso, $w:V\times V\rightarrow\mathbb{R}$, $A\subseteq E$ un sottoinsieme contenuto in un qualche albero di copertura minimo per $G$, 
$(S, V-S)$ un qualunque taglio che rispetta $A$, sia $[u, v]$ un arco leggero che attraversa il taglio, allora l'arco $[u, v]$ \`e sicuro per $A$. 
\paragraph{Dimostrazione}
Sia $T$ un albero di copertura minimo che contiene $A$, se $(u, v)\in T$ allora $(u, v)$ \`e sicuro per $A$, se $(u, v)\not\in T$ si trasforma $T$ in $T'$ che contiene $(u, v)$ e si 
dimostra che $T'$ \`e un albero di copertura minimo. Per definizione di albero $u$ e $v$ sono connessi da un cammino $C\subseteq T$ e stanno in lati opposti del taglio e $\exists(x, y)
\in C$ che attraversa il taglio. $T'=T-\{(x, y)\}\cup\{(u, v)\}$, $T'$ \`e un albero di copertura, $w(T')\le w(T)$ perch\`e $w(u, v)\le w(x, y)$, $w(T)\le w(T')$ in quanto $T$ \`e 
minimo.
\paragraph{Corollario}
Sia $G=(V, E)$ un grafo non orientato e connesso , $w:V\times V\rightarrow\mathbb{R}$, $A\subseteq E$ un sottoinsieme contenuto in un qualche albero di copertura minimo per $G$, $C$
una componente connessa nella foresta $G_A=(V, A)$ e $[u, v]$ un arco leggero che connette $C$ a qualche altra componente in $G_A$, allora l'arco $[u, v]$ \`e sicuro per $A$. 
\subsection{Algoritmo di Kruskal}
Si ingrandiscono sottoinsiemi disgiunti di un albero di copertura minimo connettendoli fra di loro fino ad avere l'albero complessivo. Si individua un arco sicuro scegliendo un arco
$[u, v]$ di peso minimo tra tutti gli archi che connettono due distinti alberi della foresta. L'algoritmo \`e greedy perch\`e ad ogni passo si aggiunge alla foresta un arco con il peso
minore. Per implementarlo si utilizza un struttura dati Merge-Find set.\\
\begin{algorithm}[H]
\DontPrintSemicolon
\SetKwComment{comment}{$\%$}{}

\SetKw{Int}{int}
\SetKw{Float}{float}
\SetKw{Boolean}{boolean}
\SetKw{New}{new}
\SetKw{True}{true}
\SetKw{False}{false}
\SetKw{Not}{not}
\SetKw{And}{and}
\SetKw{Or}{or}
\SetKw{Down}{down}
\SetKw{To}{to}
\SetKw{New}{new}
\SetKw{Return}{return}
\SetKw{Nil}{nil}
\SetKw{Print}{print}
\SetKw{Void}{void}

\SetKwData{Item}{Item}
\SetKwData{Tree}{Tree}
\SetKwData{PriorityQueue}{PriorityQueue}
\SetKwData{Mfset}{Mfset}
\SetKwData{Graph}{Graph}
\SetKwData{N}{n}
\SetKwData{Space}{ }
\SetKwData{Parent}{parent}
\SetKwData{Rank}{rank}
\SetKwData{Set}{Set}
\SetKwData{List}{List}
\SetKwData{F}{f}
\SetKwData{Left}{left}
\SetKwData{Right}{right}
\SetKwData{Mfset}{Mfset}

\SetKwFunction{Max}{max}
\SetKwFunction{Min}{min}
\SetKwFunction{Insert}{insert}
\SetKwFunction{ListCos}{List}
\SetKwFunction{Head}{head}
\SetKwFunction{Len}{len}
\SetKwFunction{SetCos}{Set}
\SetKwFunction{TreeCos}{Tree}
\SetKwFunction{MinPriorityQueue}{MinPriorityQueue}
\SetKwFunction{DeleteMin}{deleteMin}
\SetKwFunction{MfsetCos}{Mfset}
\SetKwFunction{Find}{find}
\SetKwFunction{Merge}{merge}


\SetKwProg{Fn}{}{}

\SetKwFunction{Kruskal}{kruskal}
\SetKwFunction{}{}
\SetKwFunction{}{}
\SetKwFunction{}{}
\SetKwFunction{}{}
\SetKwFunction{}{}

\SetKwData{Edge}{Edge}
\caption{\protect\Set \protect\Kruskal{\protect\Edge[] A, \protect\Int n, \protect\Int m}}
\Set T = \SetCos{}\;
\Mfset M = \MfsetCos{n}\;
$\{$ordina $A[1, \dots, m]$ in modo che $A[1].weight\le\cdots\le A[m].weight\}$\;
\Int count = 0;
\Int i = 1;
\comment{Termina quando l'albero ha $n-1$ archi o non ci sono pi\`u archi}
\While{count $<$ n - 1 \And i $\le$ m}{
	\If{M.\Find{A[i].u} $\neq$ M.\Find{A[i].v}}{
		M.\Merge{A[i].u, A[i].v}\;
		T.\Insert{A[i]}\;
		count += 1\;
	}
	i += 1\;
}
\Return T\;
\end{algorithm}

\subsubsection{Analisi della complessit\`a}
Il tempo di esecuzione per l'algoritmo di Kruskal dipende dalla realizzazione della struttura dati per Merge-Find Set, utilizzando la versione con euristica sul rango e compressione le
cui operazioni hanno costo ammortizzato costante si ha:
\begin{center}
\begin{tabular}{|c|c|c|}
	\hline
	\textbf{Fase} & \textbf{Volte} & \textbf{Costo} \\
	\hline
	Inizializzazione & $1$ & $O(n)$ \\
	\hline
	Ordinamento & $1$ & $O(m\log m)$ \\
	\hline
	Operazioni \emph{find()}, \emph{merge()} & $O(m)$ & $O(1)$\\
	\hline
\end{tabular}
\end{center}
Il totale \`e $O(n+m\log m + m) = O(m\log m) = O(m\log n^2) = O(m\log n)$.
\subsection{Algoritmo di Prim}
Si mantiene in $A$ un singolo albero che parte da un vertice arbitrario $r$ e cresce fino a che non ricopre tutti i vertici. Ad ogni passo viene aggiunto un arco leggero che collega
un vertice in $V_A$ con un vertice in $V-V_A$.
\paragraph{Correttezza} Per definizione $(V_A, V - V_A)$ \`e un taglio che rispetta $A$ e per il corollario gli archi leggeri che attraversano il taglio sono sicuri. 
\subsubsection{Implementazione}
Durante l'esecuzione i vertici non ancora nell'albero si trovano in una coda con min-priorit\`a $Q$ ordinata secondo la priorit\`a: la priorit\`a del nodo $v$ \`e il peso minimo di un
arco che collega $v$ ad un vertice nell'albero o $+\infty$ se tale arco non esiste. L'albero viene mantenuto come vettore dei padri. $A$ \`e mantenuto implicitamente: $A=\{[v, p[v]]|v\in
V-Q-\{r\}\}$. \\
\begin{algorithm}[H]
\DontPrintSemicolon
\SetKwComment{comment}{$\%$}{}

\SetKw{Int}{int}
\SetKw{Float}{float}
\SetKw{Boolean}{boolean}
\SetKw{New}{new}
\SetKw{True}{true}
\SetKw{False}{false}
\SetKw{Not}{not}
\SetKw{And}{and}
\SetKw{Or}{or}
\SetKw{Down}{down}
\SetKw{To}{to}
\SetKw{New}{new}
\SetKw{Return}{return}
\SetKw{Nil}{nil}
\SetKw{Print}{print}
\SetKw{Void}{void}

\SetKwData{Item}{Item}
\SetKwData{Tree}{Tree}
\SetKwData{PriorityQueue}{PriorityQueue}
\SetKwData{Mfset}{Mfset}
\SetKwData{Graph}{Graph}
\SetKwData{N}{n}
\SetKwData{Space}{ }
\SetKwData{Parent}{parent}
\SetKwData{Rank}{rank}
\SetKwData{Set}{Set}
\SetKwData{List}{List}
\SetKwData{F}{f}
\SetKwData{Left}{left}
\SetKwData{Right}{right}
\SetKwData{Mfset}{Mfset}
\SetKwData{PriorityItem}{PriorityItem}
\SetKwData{Node}{Node}

\SetKwFunction{Max}{max}
\SetKwFunction{Min}{min}
\SetKwFunction{Insert}{insert}
\SetKwFunction{ListCos}{List}
\SetKwFunction{Head}{head}
\SetKwFunction{Len}{len}
\SetKwFunction{SetCos}{Set}
\SetKwFunction{TreeCos}{Tree}
\SetKwFunction{MinPriorityQueue}{MinPriorityQueue}
\SetKwFunction{DeleteMin}{deleteMin}
\SetKwFunction{MfsetCos}{Mfset}
\SetKwFunction{Find}{find}
\SetKwFunction{Merge}{merge}
\SetKwFunction{IsEmpty}{isEmpty}
\SetKwFunction{Adj}{adj}
\SetKwFunction{Decrease}{decrease}


\SetKwProg{Fn}{}{}

\SetKwFunction{Prim}{prim}
\SetKwFunction{}{}
\SetKwFunction{}{}
\SetKwFunction{}{}
\SetKwFunction{}{}
\SetKwFunction{}{}

\SetKwData{Edge}{Edge}
\caption{\protect\Prim{\protect\Graph G, \protect\Int[] p}}
\PriorityQueue Q = \MinPriorityQueue{}\;
\PriorityItem[] pos = \New \PriorityItem$[1\dots G.n]$\;
\ForEach{u $\in$ G.V() - $\{r\}$}{
	pos[u] = Q.\Insert{u, $+\infty$}\;
}
pos[r] = Q.\Insert{r, 0}\;
p[r] = 0\;
\While{\Not Q.\IsEmpty{}}{
	\Node u = Q.\DeleteMin{}\;
	pos[u] = \Nil\;
	\ForEach{v $\in$ G.\Adj{u}}{
		\If{pos[v] $\neq$ \Nil \And w(u, v) $<$ pos[v].priority}{
			Q.\Decrease{pos[v], w(u, v)}\;
			p[v] = u\;
		}
	}
}
\end{algorithm}

\subsubsection{Analisi complessit\`a}
L'efficienza dell'algoritmo di Prim dipende dall'implementazione della coda di priorit\`a. Se si utilizza uno heap binario si ha:
\begin{center}
\begin{tabular}{|c|c|c|}
	\hline
	\textbf{Fase} & \textbf{Volte} & \textbf{Costo} \\
	\hline
	Inizializzazione & $1$ & $O(n\log n)$ \\
	\hline
	\emph{deleteMin()} & $O(n)$ & $O(\log m)$ \\
	\hline
	\emph{decreasePriority()} & $O(m)$ & $O(\log n)$\\
	\hline
\end{tabular}
\end{center}
Per un tempo totale $O(n+n\log n+m\log n) = O(m\log n)$, asintoticamente uguale a quello di Kruskal. Le cose cambiano se si utilizza un vettore non ordinato.
