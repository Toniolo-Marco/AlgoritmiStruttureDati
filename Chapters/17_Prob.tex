\chapter{Algoritmi Probabilistici}
La filosofia degli algoritmi probabilistici \`e quella di fare una scelta casuale se non si sa quale strada prendere. Fino ad ora la casualit\`a si \`e vista nell'analisi del caso medio
quando si calcola la media di tutti i possibili dati di ingresso in base ad una distribuzione di probabilit\`a e nell'hashing in quanto le funzioni hash equivalgono a randomizzare le
chiavi secondo una distribuzione uniforme. Negli algoritmi probabilistici il calcolo delle probabilit\`a \`e applicato ai dati di output e si dividono in:
\begin{itemize}
	\item Algoritmi corretti il cui tempo di funzionamento \`e probabilistico (Las Vegas).
	\item Algoritmi la cui correttezza \`e probabilistica (Montecarlo).
\end{itemize}
\section{Algoritmi Montecarlo}
\subsection{Test di primalit\`a}
\subsubsection{Piccolo teorema di Fermat}
Per ogni numero primo $n$ e ogni $b\in [2,\dots, n - 1]: b^{n-1}\mod n = 1$. 
\paragraph{Test di primalit\`a di Fermat}\mbox{}\\
\begin{multicols}{2}
\begin{algorithm}[H]
\DontPrintSemicolon
\SetKwComment{comment}{$\%$}{}
\SetKwRepeat{Do}{do}{while}
\SetKw{Int}{int}
\SetKw{Float}{float}
\SetKw{Boolean}{boolean}
\SetKw{New}{new}
\SetKw{True}{true}
\SetKw{False}{false}
\SetKw{Not}{not}
\SetKw{And}{and}
\SetKw{Or}{or}
\SetKw{Down}{down}
\SetKw{To}{to}
\SetKw{New}{new}
\SetKw{Return}{return}
\SetKw{Nil}{nil}
\SetKw{Print}{print}
\SetKw{Void}{void}

\SetKwData{Item}{Item}
\SetKwData{Tree}{Tree}
\SetKwData{PriorityQueue}{PriorityQueue}
\SetKwData{Mfset}{Mfset}
\SetKwData{Graph}{Graph}
\SetKwData{N}{n}
\SetKwData{Space}{ }
\SetKwData{Parent}{parent}
\SetKwData{Rank}{rank}
\SetKwData{Set}{Set}
\SetKwData{List}{List}
\SetKwData{F}{f}
\SetKwData{Left}{left}
\SetKwData{Right}{right}
\SetKwData{Dictionary}{Dictionary}
\SetKwData{Mfset}{Mfset}
\SetKwData{PriorityItem}{PriorityItem}
\SetKwData{Node}{Node}
\SetKwData{Point}{Point}
\SetKwData{Stack}{Stack}

\SetKwFunction{Max}{max}
\SetKwFunction{Min}{min}
\SetKwFunction{Insert}{insert}
\SetKwFunction{ListCos}{List}
\SetKwFunction{Head}{head}
\SetKwFunction{Len}{len}
\SetKwFunction{SetCos}{Set}
\SetKwFunction{TreeCos}{Tree}
\SetKwFunction{MinPriorityQueue}{MinPriorityQueue}
\SetKwFunction{DeleteMin}{deleteMin}
\SetKwFunction{MfsetCos}{Mfset}
\SetKwFunction{Find}{find}
\SetKwFunction{Lookup}{lookup}
\SetKwFunction{Merge}{merge}
\SetKwFunction{IsEmpty}{isEmpty}
\SetKwFunction{Adj}{adj}
\SetKwFunction{Push}{push}
\SetKwFunction{Pop}{pop}
\SetKwFunction{Top}{top}
\SetKwFunction{Decrease}{decrease}
\SetKwFunction{StackCos}{StackCos}


\SetKwProg{Fn}{}{}

\SetKwFunction{IsPrime}{isPrime1}
\SetKwFunction{Random}{random}
\SetKwFunction{}{}
\SetKwFunction{}{}
\SetKwFunction{}{}
\SetKwFunction{}{}

\caption{\protect\Boolean \protect\IsPrime{\protect\Int n}}
b = \Random{2, n - 1}\;
\If{$b^{n-1}\mod n \neq 1$}{
	\Return \False\;
}
\Return \True\;

\end{algorithm}

Si noti come esistono numeri composti tali che $\exists b\in[2, \dots, n - 1]: b^{n-1}\mod n = 1$, pertanto:
\begin{itemize}
	\item Output \textbf{false}: sicuramente composto.
	\item Output \textbf{true}: possibilmente primo.
\end{itemize}
\columnbreak
\begin{algorithm}[H]
\DontPrintSemicolon
\SetKwComment{comment}{$\%$}{}
\SetKwRepeat{Do}{do}{while}
\SetKw{Int}{int}
\SetKw{Float}{float}
\SetKw{Boolean}{boolean}
\SetKw{New}{new}
\SetKw{True}{true}
\SetKw{False}{false}
\SetKw{Not}{not}
\SetKw{And}{and}
\SetKw{Or}{or}
\SetKw{Down}{down}
\SetKw{To}{to}
\SetKw{New}{new}
\SetKw{Return}{return}
\SetKw{Nil}{nil}
\SetKw{Print}{print}
\SetKw{Void}{void}

\SetKwData{Item}{Item}
\SetKwData{Tree}{Tree}
\SetKwData{PriorityQueue}{PriorityQueue}
\SetKwData{Mfset}{Mfset}
\SetKwData{Graph}{Graph}
\SetKwData{N}{n}
\SetKwData{Space}{ }
\SetKwData{Parent}{parent}
\SetKwData{Rank}{rank}
\SetKwData{Set}{Set}
\SetKwData{List}{List}
\SetKwData{F}{f}
\SetKwData{Left}{left}
\SetKwData{Right}{right}
\SetKwData{Dictionary}{Dictionary}
\SetKwData{Mfset}{Mfset}
\SetKwData{PriorityItem}{PriorityItem}
\SetKwData{Node}{Node}
\SetKwData{Point}{Point}
\SetKwData{Stack}{Stack}

\SetKwFunction{Max}{max}
\SetKwFunction{Min}{min}
\SetKwFunction{Insert}{insert}
\SetKwFunction{ListCos}{List}
\SetKwFunction{Head}{head}
\SetKwFunction{Len}{len}
\SetKwFunction{SetCos}{Set}
\SetKwFunction{TreeCos}{Tree}
\SetKwFunction{MinPriorityQueue}{MinPriorityQueue}
\SetKwFunction{DeleteMin}{deleteMin}
\SetKwFunction{MfsetCos}{Mfset}
\SetKwFunction{Find}{find}
\SetKwFunction{Lookup}{lookup}
\SetKwFunction{Merge}{merge}
\SetKwFunction{IsEmpty}{isEmpty}
\SetKwFunction{Adj}{adj}
\SetKwFunction{Push}{push}
\SetKwFunction{Pop}{pop}
\SetKwFunction{Top}{top}
\SetKwFunction{Decrease}{decrease}
\SetKwFunction{StackCos}{StackCos}


\SetKwProg{Fn}{}{}

\SetKwFunction{IsPrime}{isPrime2}
\SetKwFunction{Random}{random}
\SetKwFunction{}{}
\SetKwFunction{}{}
\SetKwFunction{}{}
\SetKwFunction{}{}

\caption{\protect\Boolean \protect\IsPrime{\protect\Int n}}
\For{\Int i = 1 \To k}{
	b = \Random{2, n - 1}\;
	\If{$b^{n-1}\mod n \neq 1$}{
		\Return \False\;
	}
}
\Return \True\;
\end{algorithm}

Si noti come esistono numeri composti (numeri di Carmichael) tali che $\forall b\in[2, \dots, n - 1]: b^{n-1}\mod n = 1$, pertanto:
\begin{itemize}
	\item Output \textbf{false}: sicuramente composto.
	\item Output \textbf{true}: possibilmente primo.
\end{itemize}
\end{multicols}
\subsubsection{Test di Miller-Rabin}
Si esprima $n-1$ come $n-1 = m2^v$ con $m$ dispari, la rappresentazione binaria di $n-1$ \`e uguale alla rappresentazione binaria del numero dispari $m$ seguito da $v$ zeri. Sia $b$
un numero in $[1, \dots, n - 1]$, e se un numero $n$ \`e primo allora $mcd(n, b) = 1$ e $b^m\mod n = 1\lor \exists i, 0\le i \le v -1: b^{2^im}\mod n = 1$, mentre se \`e composto almeno
una delle due condizioni \`e falsa. Rabin ha dimostrato che se $n$ \`e composto allora ci sono almeno $\frac{3}{4}(n - 1)$ valori in $[1, \dots, n - 1]$ per i quali una delle condizioni
\`e falsa e pertanto il test di compostezza ha una probabilit\`a inferiore a $\frac{1}{4}$ di rispondere erroneamente. 
\begin{algorithm}[H]
\DontPrintSemicolon
\SetKwComment{comment}{$\%$}{}
\SetKwRepeat{Do}{do}{while}
\SetKw{Int}{int}
\SetKw{Float}{float}
\SetKw{Boolean}{boolean}
\SetKw{New}{new}
\SetKw{True}{true}
\SetKw{False}{false}
\SetKw{Not}{not}
\SetKw{And}{and}
\SetKw{Or}{or}
\SetKw{Down}{down}
\SetKw{To}{to}
\SetKw{New}{new}
\SetKw{Return}{return}
\SetKw{Nil}{nil}
\SetKw{Print}{print}
\SetKw{Void}{void}

\SetKwData{Item}{Item}
\SetKwData{Tree}{Tree}
\SetKwData{PriorityQueue}{PriorityQueue}
\SetKwData{Mfset}{Mfset}
\SetKwData{Graph}{Graph}
\SetKwData{N}{n}
\SetKwData{Space}{ }
\SetKwData{Parent}{parent}
\SetKwData{Rank}{rank}
\SetKwData{Set}{Set}
\SetKwData{List}{List}
\SetKwData{F}{f}
\SetKwData{Left}{left}
\SetKwData{Right}{right}
\SetKwData{Dictionary}{Dictionary}
\SetKwData{Mfset}{Mfset}
\SetKwData{PriorityItem}{PriorityItem}
\SetKwData{Node}{Node}
\SetKwData{Point}{Point}
\SetKwData{Stack}{Stack}

\SetKwFunction{Max}{max}
\SetKwFunction{Min}{min}
\SetKwFunction{Insert}{insert}
\SetKwFunction{ListCos}{List}
\SetKwFunction{Head}{head}
\SetKwFunction{Len}{len}
\SetKwFunction{SetCos}{Set}
\SetKwFunction{TreeCos}{Tree}
\SetKwFunction{MinPriorityQueue}{MinPriorityQueue}
\SetKwFunction{DeleteMin}{deleteMin}
\SetKwFunction{MfsetCos}{Mfset}
\SetKwFunction{Find}{find}
\SetKwFunction{Lookup}{lookup}
\SetKwFunction{Merge}{merge}
\SetKwFunction{IsEmpty}{isEmpty}
\SetKwFunction{Adj}{adj}
\SetKwFunction{Push}{push}
\SetKwFunction{Pop}{pop}
\SetKwFunction{Top}{top}
\SetKwFunction{Decrease}{decrease}
\SetKwFunction{StackCos}{StackCos}


\SetKwProg{Fn}{}{}

\SetKwFunction{IsPrime}{isPrime}
\SetKwFunction{Random}{random}
\SetKwFunction{IsComposite}{isComposite}
\SetKwFunction{}{}
\SetKwFunction{}{}
\SetKwFunction{}{}

\caption{\protect\Boolean \protect\IsPrime{\protect\Int n}}
\For{\Int i = 1 \To k}{
	\Int b = \Random{2, n -1}\;
	\If{\IsComposite{n, b}}{
		\Return False\;
	}
}
\Return \True\;
\end{algorithm}

Questo algoritmo ha complessit\`a $O(k\log^2 n\log\log n\log\log\log n)$ con probabilit\`a d'errore $\bigl(\frac{1}{4}\bigr)^k$
\subsection{Espressione polinomiale nulla}
Data un'espressione algebrica polinomiale $p(x_1, \dots, x_n)$ in $n$ variabili si deve determinare se $p$ \`e identicamente nulla oppure no. Si assuma che non sia in forma di monomi. Si
noti come gli algoritmi basati su semplificazioni sono molto complessi. 
\subsubsection{Algoritmo}
Si genera una ennupla di valori $v_1, \dots, v_n$, si calcola $x = p(v_1, \dots, v_n)$, se $x\neq 0$ $p$ non \`e identicamente nulla, altrimenti potrebbe essere identicamente nulla. Se 
ogni $v_i$ \`e un valore intero compreso causale fra $1$ e $2d$, dove $d$ \`e il grado del polinomio allora la probabilit\`a di errore non supera $\frac{1}{2}$. Il processo si ripete
$k$ volte, riducendo la probabilit\`a d'errore a $\bigl(\frac{1}{2}\bigr)^k$.
\section{Algoritmi Las Vegas}
Gli algoritmi statistici sui vettori estraggono alcune caratteristiche statisticamente rilevanti da un vettore numerico come:
\begin{itemize}
	\item Media: $\mu = \frac{1}{n}\sum\limits_{i = 1}^n A[i]$.
	\item Varianza: $\sigma^2 \frac{1}{n}\sum\limits_{i = 1}^n(A[i] - \mu)^2$.
	\item Moda: il valore o i valori pi\`u frequenti. 
\end{itemize}
\subsection{Statistiche d'ordine}
Dato un array $A$ contenente $n$ valori e un valore $1\le k \le n$, si deve trovare l'elemento che occuperebbe la posizione $k$ se il vettore fosse ordinato. Il problema del calcolo 
della mediana \`e un sottoproblema del problema della selezione con $k = \bigl\lceil\frac{n}{2}\bigr\rceil$.
\subsubsection{Selezione per piccoli valori di $\mathbf{k}$}
Si pu\`o utilizzare uno heap e l'algoritmo pu\`o essere generalizzato a valori generici di $k>2$. 
\begin{multicols}{2}
\begin{algorithm}[H]
\DontPrintSemicolon
\SetKwComment{comment}{$\%$}{}
\SetKwRepeat{Do}{do}{while}
\SetKw{Int}{int}
\SetKw{Float}{float}
\SetKw{Boolean}{boolean}
\SetKw{New}{new}
\SetKw{True}{true}
\SetKw{False}{false}
\SetKw{Not}{not}
\SetKw{And}{and}
\SetKw{Or}{or}
\SetKw{Down}{down}
\SetKw{To}{to}
\SetKw{New}{new}
\SetKw{Return}{return}
\SetKw{Nil}{nil}
\SetKw{Print}{print}
\SetKw{Void}{void}

\SetKwData{Item}{Item}
\SetKwData{Tree}{Tree}
\SetKwData{PriorityQueue}{PriorityQueue}
\SetKwData{Mfset}{Mfset}
\SetKwData{Graph}{Graph}
\SetKwData{N}{n}
\SetKwData{Space}{ }
\SetKwData{Parent}{parent}
\SetKwData{Rank}{rank}
\SetKwData{Set}{Set}
\SetKwData{List}{List}
\SetKwData{F}{f}
\SetKwData{Left}{left}
\SetKwData{Right}{right}
\SetKwData{Dictionary}{Dictionary}
\SetKwData{Mfset}{Mfset}
\SetKwData{PriorityItem}{PriorityItem}
\SetKwData{Node}{Node}
\SetKwData{Point}{Point}
\SetKwData{Stack}{Stack}

\SetKwFunction{Max}{max}
\SetKwFunction{Min}{min}
\SetKwFunction{Insert}{insert}
\SetKwFunction{ListCos}{List}
\SetKwFunction{Head}{head}
\SetKwFunction{Len}{len}
\SetKwFunction{SetCos}{Set}
\SetKwFunction{TreeCos}{Tree}
\SetKwFunction{MinPriorityQueue}{MinPriorityQueue}
\SetKwFunction{DeleteMin}{deleteMin}
\SetKwFunction{MfsetCos}{Mfset}
\SetKwFunction{Find}{find}
\SetKwFunction{Lookup}{lookup}
\SetKwFunction{Merge}{merge}
\SetKwFunction{IsEmpty}{isEmpty}
\SetKwFunction{Adj}{adj}
\SetKwFunction{Push}{push}
\SetKwFunction{Pop}{pop}
\SetKwFunction{Top}{top}
\SetKwFunction{Decrease}{decrease}
\SetKwFunction{StackCos}{StackCos}


\SetKwProg{Fn}{}{}

\SetKwFunction{HeapSelect}{heapSelect}
\SetKwFunction{BuildHeap}{buildHeap}
\SetKwFunction{}{}
\SetKwFunction{}{}
\SetKwFunction{}{}
\SetKwFunction{}{}

\caption{\protect\Int \protect\HeapSelect{\protect\Item[] A, \protect\Int n, \protect\Int k}}
\BuildHeap{A}\;
\For{\Int i = 0 \To k - 1}{
	\DeleteMin{A, n}\;
}
\Return \DeleteMin{A, n}\;
\end{algorithm}

\columnbreak
\paragraph{Complessit\`a}\mbox{}
$O(n + k\log n)$, se $k = O\bigl(\frac{n}{\log n}\bigr)$ e se $k = \frac{n}{2}$ non va bene. 
\end{multicols}
\subsubsection{Algoritmo di selezione}
Si considera un approccio divide-et-impera simile al Quicksort ed essendo un problema di ricerca non \`e necessario cercare in entrambe le partizioni ma in una sola di esse prestando 
attenzione agli indici. 
\begin{algorithm}[H]
\DontPrintSemicolon
\SetKwComment{comment}{$\%$}{}
\SetKwRepeat{Do}{do}{while}
\SetKw{Int}{int}
\SetKw{Float}{float}
\SetKw{Boolean}{boolean}
\SetKw{New}{new}
\SetKw{True}{true}
\SetKw{False}{false}
\SetKw{Not}{not}
\SetKw{And}{and}
\SetKw{Or}{or}
\SetKw{Down}{down}
\SetKw{To}{to}
\SetKw{New}{new}
\SetKw{Return}{return}
\SetKw{Nil}{nil}
\SetKw{Print}{print}
\SetKw{Void}{void}

\SetKwData{Item}{Item}
\SetKwData{Tree}{Tree}
\SetKwData{PriorityQueue}{PriorityQueue}
\SetKwData{Mfset}{Mfset}
\SetKwData{Graph}{Graph}
\SetKwData{N}{n}
\SetKwData{Space}{ }
\SetKwData{Parent}{parent}
\SetKwData{Rank}{rank}
\SetKwData{Set}{Set}
\SetKwData{List}{List}
\SetKwData{F}{f}
\SetKwData{Left}{left}
\SetKwData{Right}{right}
\SetKwData{Dictionary}{Dictionary}
\SetKwData{Mfset}{Mfset}
\SetKwData{PriorityItem}{PriorityItem}
\SetKwData{Node}{Node}
\SetKwData{Point}{Point}
\SetKwData{Stack}{Stack}

\SetKwFunction{Max}{max}
\SetKwFunction{Min}{min}
\SetKwFunction{Insert}{insert}
\SetKwFunction{ListCos}{List}
\SetKwFunction{Head}{head}
\SetKwFunction{Len}{len}
\SetKwFunction{SetCos}{Set}
\SetKwFunction{TreeCos}{Tree}
\SetKwFunction{MinPriorityQueue}{MinPriorityQueue}
\SetKwFunction{DeleteMin}{deleteMin}
\SetKwFunction{MfsetCos}{Mfset}
\SetKwFunction{Find}{find}
\SetKwFunction{Lookup}{lookup}
\SetKwFunction{Merge}{merge}
\SetKwFunction{IsEmpty}{isEmpty}
\SetKwFunction{Adj}{adj}
\SetKwFunction{Push}{push}
\SetKwFunction{Pop}{pop}
\SetKwFunction{Top}{top}
\SetKwFunction{Decrease}{decrease}
\SetKwFunction{StackCos}{StackCos}


\SetKwProg{Fn}{}{}

\SetKwFunction{Selection}{selection}
\SetKwFunction{Pivot}{pivot}
\SetKwFunction{}{}
\SetKwFunction{}{}
\SetKwFunction{}{}
\SetKwFunction{}{}

\caption{\protect\Item \protect\Selection{\protect\Item[] A, \protect\Int start, \protect\Int end, \protect\Int k}}
\uIf{start == end}{
	\Return A[start]\;
}
\Else{
	\Int j = \Pivot{A, start, end}\;
	\Int q = j - start + 1\;
	\uIf{k == q}{
		\Return A[j]\;
	}
	\uElseIf{k $<$ q}{
		\Return \Selection{A, start, j - 1, k}\;
	}
	\Else{
		\Return \Selection{A, j + 1, end, k - q}\;
	}
}
\end{algorithm}

\paragraph{Complessit\`a}
\subparagraph{Caso pessimo}
$$T(n) = 
\begin{cases}
	1 & n\le 1\\
	T(n - 1) + n > 1\\
\end{cases}
$$
$T(n) = O(n^2)$.
\subparagraph{Caso ottimo}
$$T(n) = 
\begin{cases}
	1 & n\le 1\\
	T\bigl(\frac{n}{2}\bigr) + n > 1\\
\end{cases}
$$
$T(n) = O(n)$.
\subparagraph{Caso medio}\mbox{}
Si assuma che \emph{pivot()} restituisca con la stessa probabilit\`a una qualsiasi posizione $j$ del vettore $A$.
\begin{align*}
	T(n) &= n - 1 + \frac{1}{n}\sum\limits_{q = 1}^n T(\max\{q - 1, n - q\}) & \text{Media su } n \text{ casi}\\
	     &\le n - 1 + \frac{1}{n}\sum\limits_{q = \lceil\frac{n}{2}\rceil}^{n-1}2T(q) & \text{Per } n > 1\\
\end{align*}
\begin{align*}
	T(n) &\le n + \frac{1}{n}\sum\limits_{q = \lceil\frac{n}{2}\rceil}^{n-1}2\cdot cq\le n + \frac{2c}{n}\sum\limits_{q = \lceil\frac{n}{2}\rceil}^{n-1}q &\text{Sostituzione, raccolgo } 2c\\
	     &\le n + \frac{2c}{n}\sum\limits_{q = \frac{n}{2}-1}^{n-1}q &\text{Estensione sommatoria}\\
	     &= n + \frac{2c}{n}\biggl(\sum\limits_{q = 1}^{n-1} q - \sum\limits_{q = 1}^{\frac{n}{2}-2} q\biggr) &\text{Sottrazione prima parte}\\
	     &= n + \frac{2c}{n}\cdot\biggl(\dfrac{n(n-1)}{2}-\dfrac{(\frac{n}{2} - 1)(\frac{n}{2} - 2)}{2}\biggr)\\
	     &= n + \frac{\not2c}{n}\cdot\dfrac{(n^2 - n - (\frac{1}{4}n^2-\frac{3}{2}n + 2)}{\not2}\\
	     &= n + \frac{c}{n}\cdot\biggl(\frac{3}{4}n^2-\frac{1}{2}n-2\biggr)\\
	     &\le n + \frac{c}{n}\cdot\biggl(\frac{3}{4}n^2\biggr) = n + \frac{3}{4}cn \le cn &\text{Vera per }c\le 4\\
\end{align*}
Si \`e partiti dall'assunzione che $j$ assume equiprobabilisticamente tutti i valori compresi tra $1$ e $n$ e pertanto questo concetto va forzato : $A[random(start,\ end)]\leftrightarrow
A[start]$, accorgimento che vale anche per \emph{QuickSort} con complessit\`a nel caso medio $O(n)$ per la soluzione e $O(n\log n)$ per l'ordinamento. 
\subsubsection{Selezione deterministica}
Si supponga di avere un algoritmo black box che ritorni il mediano di $n$ valori in tempo $O(n)$, in questo modo il problema della selezione sarebbe risolto e lo si pu\`o considere
rilassando le pretese: si supponga di avere un tale algoritmo che ritorni un valore che dista al pi\`u $\frac{3}{10}n$ dal mediano nell'ordinamento in modo da utilizzarlo per ottimizzare
il problema della soluzione.
\paragraph{Algoritmo}
Si suddividano i valori in gruppi di $5$ e l'i-esimo gruppo $S_i$, con $i\in[1, \bigl\lceil\frac{n}{5}\bigr\rceil]$. Si trova il mediano $M_i$ di ogni gruppo $S_i$ e tramite una chiamata
ricorsiva si trova il mediano $m$ dei mediani $[M_1, \dots, M_{\lceil\frac{n}{5}\rceil}]$. Si usa $m$ come pivot e si richiama l'algoritmo ricorsivamente sull'array opportuno come
nella \emph{selection()} randomizzata e quando la dimensione scende sotto una certa dimensione si pu\`o utilizzare un algoritmo di ordinamento per trovare il mediano.\\
\begin{algorithm}[H]
\DontPrintSemicolon
\SetKwComment{comment}{$\%$}{}
\SetKwRepeat{Do}{do}{while}
\SetKw{Int}{int}
\SetKw{Float}{float}
\SetKw{Boolean}{boolean}
\SetKw{New}{new}
\SetKw{True}{true}
\SetKw{False}{false}
\SetKw{Not}{not}
\SetKw{And}{and}
\SetKw{Or}{or}
\SetKw{Down}{down}
\SetKw{To}{to}
\SetKw{New}{new}
\SetKw{Return}{return}
\SetKw{Nil}{nil}
\SetKw{Print}{print}
\SetKw{Void}{void}

\SetKwData{Item}{Item}
\SetKwData{Tree}{Tree}
\SetKwData{PriorityQueue}{PriorityQueue}
\SetKwData{Mfset}{Mfset}
\SetKwData{Graph}{Graph}
\SetKwData{N}{n}
\SetKwData{Space}{ }
\SetKwData{Parent}{parent}
\SetKwData{Rank}{rank}
\SetKwData{Set}{Set}
\SetKwData{List}{List}
\SetKwData{F}{f}
\SetKwData{Left}{left}
\SetKwData{Right}{right}
\SetKwData{Dictionary}{Dictionary}
\SetKwData{Mfset}{Mfset}
\SetKwData{PriorityItem}{PriorityItem}
\SetKwData{Node}{Node}
\SetKwData{Point}{Point}
\SetKwData{Stack}{Stack}

\SetKwFunction{Max}{max}
\SetKwFunction{Min}{min}
\SetKwFunction{Insert}{insert}
\SetKwFunction{ListCos}{List}
\SetKwFunction{Head}{head}
\SetKwFunction{Len}{len}
\SetKwFunction{SetCos}{Set}
\SetKwFunction{TreeCos}{Tree}
\SetKwFunction{MinPriorityQueue}{MinPriorityQueue}
\SetKwFunction{DeleteMin}{deleteMin}
\SetKwFunction{MfsetCos}{Mfset}
\SetKwFunction{Find}{find}
\SetKwFunction{Lookup}{lookup}
\SetKwFunction{Merge}{merge}
\SetKwFunction{IsEmpty}{isEmpty}
\SetKwFunction{Adj}{adj}
\SetKwFunction{Push}{push}
\SetKwFunction{Pop}{pop}
\SetKwFunction{Top}{top}
\SetKwFunction{Decrease}{decrease}
\SetKwFunction{StackCos}{StackCos}


\SetKwProg{Fn}{}{}

\SetKwFunction{Selection}{select}
\SetKwFunction{Pivot}{pivot}
\SetKwFunction{InsertionrSort}{insertionSort}
\SetKwFunction{Median}{median5}
\SetKwFunction{}{}
\SetKwFunction{}{}

\caption{\protect\Item \protect\Selection{\protect\Item[] A, \protect\Int start, \protect\Int end, \protect\Int k}}
\comment{Se la dimensione \`e inferiore ad una soglia (10) ordina il vettore e}
\comment{restituisci il $k$-esimo elemento di $A[start\dots end]$}
\If{end - start + 1 $\le$ 10}{
	\InsertionSort{A, start, end} \comment{Versione con indici inizio/fine}
	\Return A[start + k - 1]\;
}
\comment{Divide $A$ in $\biggl\lceil\frac{n}{5}\biggr\rceil$ sottovettori di dimensione $5$ e ne calcola la mediana}
M = \New \Int$\biggl[1\dots \biggl\lceil\frac{n}{5}\biggr\rceil\biggr]$\;
\For{\Int i = 1 \To $\biggl\lceil\frac{n}{5}\biggr\rceil$}{
	M[i] = \Median{A, start + $(i - 1)\cdot 5$, end}\;
}
\comment{Individua la mediana delle mediane e la si usa come perno}
\Item m = \Selection{M, 1, $\biggl\lceil\frac{n}{5}\biggr\rceil$, $\biggl\lceil\frac{\lceil\frac{n}{5}\rceil}{2}\biggr\rceil$}\;
\Int j = \Pivot{A, start, end,  m}\;
\comment{Calcola l'indice $q$ di $m$ in $[start\dots end]$}
\comment{Confronta $q$ con l'indice cercato e ritorna il valore conseguente}
\Int q = j - start + 1\;
\uIf{q == k}{
	\Return m\;
}
\uElseIf{q $<$ k}{
	\Return \Selection{A, start, q - 1, k}\;
}
\Else{
	\Return \Selection{A, q + 1, end, k - q}\;
}
\end{algorithm}

Si noti come l'algoritmo \emph{select()} esegue nel caso pessimo $O(n)$ confronti.
