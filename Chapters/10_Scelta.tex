\chapter{Scelta}
\section{Il problema dei cammini minimi}
\paragraph{Definizione}
Dato un cammino $p=<v_1, v_2, \dots, v_k>$ con $k>1$ il costo del cammino \`e dato da $$w(p) = \sum\limits_{i =2}^k w(v_{i-1}, v_i)$$
\paragraph{Input}
Un grafo orientato $G = (V, E)$, un nodo sorgente $s$ e una funzione di peso $w: E\rightarrow \mathbb{R}$.
\paragraph{Output}
Trovare un cammino da $s$ a $u$ per ogni nodo $u\in V$, il cui costo sia minimo.
\subsection{Varianti del problema}
\subsubsection{Cammini minimi da sorgente unica}
\subparagraph{Input:} un grafo pesato con nodo radice $s$.
\subparagraph{Output:} i cammini minimi che vanno da $s$ a tutti gli altri nodi.
\subsubsection{Cammino minimo tra una coppia di vertici}
\subparagraph{Input:} un grafo pesato e una coppia di vertici $s$, $d$.
\subparagraph{Output:} un cammino minimo fra $s$ e $d$, per risolverlo si risolve il problema dei cammini minimi da sorgente unica e si estrae il cammino richiesto in quanto non si 
conoscono algoritmi con tempo di esecuzione migliore.
\subsubsection{Cammini minimi tra tutte le coppie di vertici}
\subparagraph{Input:} un grafo pesato.
\subparagraph{Output:} i cammini minimi fra tutte le coppie di vertici, la soluzione si basa su programmazione dinamica.
\subsubsection{Pesi}
Alcuni algoritmi possono funzionare unicamente con alcune categorie speciali di pesi: positivi o sia positivi che negativi e reali o interi.
\subsection{Sottostruttura ottima}
Si noti come due cammini possono avere un tratto in comune $A\rightsquigarrow B$ ma non possono convergere in un nodo comune $B$ dopo aver percorso un tratto iniziale distinto. 
\subsubsection{Albero dei cammini minimi}
L'albero dei cammini minimi \`e un albero di copertura radicato in $s$ avente un cammino da $s$ a tutti i nodi raggiungibili da $s$. Viene detto anche spanning tree. Dato un grafo $G=
(V, E)$ non orientato e connesso, un albero di copertura di $G$ \`e un sottografo $T=(V, E_T)$ tale che $T$ \`e un albero, $E_T\subseteq E$ e $T$ contiene tutti i vertici di $G$. 
\paragraph{Rappresentazione albero}
Per rappresentare l'albero si utilizza la rappresentazione basata sul vettore dei padri.
\subsubsection{Soluzione ammissibile}
Una soluzione ammissibile pu\`o essere descritta da un albero di copertura $T$ radicato in $s$ e da un vettore di distanza $d$, i cui valori $d[u]$ rappresentano il costo del cammino da
$s$ a $u$ in $T$.\\
\begin{algorithm}[H]
\DontPrintSemicolon
\SetKwComment{comment}{$\%$}{}

\SetKw{Int}{int}
\SetKw{Boolean}{boolean}
\SetKw{New}{new}
\SetKw{True}{true}
\SetKw{False}{false}
\SetKw{Not}{not}
\SetKw{And}{and}
\SetKw{Or}{or}
\SetKw{Down}{down}
\SetKw{To}{to}
\SetKw{New}{new}
\SetKw{Return}{return}
\SetKw{Nil}{nil}
\SetKw{Print}{print}
\SetKw{Void}{void}

\SetKwData{Item}{Item}
\SetKwData{Mfset}{Mfset}
\SetKwData{Graph}{Graph}
\SetKwData{N}{n}
\SetKwData{Space}{ }
\SetKwData{Parent}{parent}
\SetKwData{Rank}{rank}
\SetKwData{Set}{Set}
\SetKwData{List}{List}
\SetKwData{Node}{Node}

\SetKwFunction{Max}{max}
\SetKwFunction{Insert}{insert}
\SetKwFunction{ListCos}{List}
\SetKwFunction{Head}{head}
\SetKwFunction{Len}{len}
\SetKwFunction{SetCos}{Set}


\SetKwProg{Fn}{}{}

\SetKwFunction{PrintPath}{printPath}
\SetKwFunction{}{}
\SetKwFunction{}{}
\SetKwFunction{}{}
\SetKwFunction{}{}
\SetKwFunction{}{}

\SetKw{Matrix}{matrix}

\caption{\protect\Void \protect\PrintPath{\protect\Node s, \protect\Node d, \protect\Node[] T}}
\uIf{s == d}{
	\Print s\;
}
\uElseIf{ T[d] == \Nil}{
	\Print "error"\;
}
\Else{
	\PrintPath{s, T[d], T}\;
	\Print d\;

}
\end{algorithm}

\section{Teorema di Bellman}
\paragraph{Enunciato}
Una soluzione ammissibile $T$ \`e ottima se e solo se: $$d[v] = d[u]+w(u, v) \quad\quad \text{per ogni arco }(u, v)\in T$$ $$d[v] \le d[u]+w(u, v) \quad\quad \text{per ogni arco }(u, v)
\in E$$
\paragraph{Dimostrazione}
\subparagraph{Se $T$ \`e una soluzione ottima, allora valgono le condizioni di Bellman} Sia $T$ una soluzione ottima e $(u, v)\in E$. Se $(u, v)\in T$ allora $d[v]=d[u]+w(u, v)$. Se
$(u, v)\not\in T$ allora $d[v]\le d[u]+w(u, v)$ perch\`e altrimenti esisterebbe nel grafo $G$ un cammino da $s$ a $v$ pi\`u corto di quello in $T$, che \`e un assurdo.
\subparagraph{Se valgono le condizioni di Bellman allora $T$ \`e una soluzione ottima} Si supponga per assurdo che il cammino da $s$ a $u$ non sia ottimo, allora esiste un albero $T'$
in cui il cammino da $s$ a $u$ ha distanza $d'[u]<d[u]$ con $d'$ il vettore delle distanze associato a $T'$. Essendo $d'[s]=d[s]=0$ ma $d'[u]<d[u]$ esiste un arco $(h, k)$ per cui 
$d'[h]\ge d[h]$ e $d'[k]<d[k]$. Si noti ora come per costruzione $d'[h]\ge d[h]$ e $d'[k]=d'[h]+w(h, k)$, e per ipotesi $d[k]\le d[h]+w(h, k)$. Combinando queste due relazioni si ottiene
$$ d'[k] = d'[h]+w(h, k)\ge d[h]+w(h, k)\ge d[k]$$ Pertanto $d'[k]\ge d[k]$, che contraddice $d'[k]<d[k]$. 
\subsection{Implementazioni}
\subsubsection{Algoritmo prototipo}
\begin{algorithm}[H]
\DontPrintSemicolon
\SetKwComment{comment}{$\%$}{}

\SetKw{Int}{int}
\SetKw{Boolean}{boolean}
\SetKw{New}{new}
\SetKw{True}{true}
\SetKw{False}{false}
\SetKw{Not}{not}
\SetKw{And}{and}
\SetKw{Or}{or}
\SetKw{Down}{down}
\SetKw{To}{to}
\SetKw{New}{new}
\SetKw{Return}{return}
\SetKw{Nil}{nil}
\SetKw{Print}{print}
\SetKw{Void}{void}

\SetKwData{Item}{Item}
\SetKwData{Mfset}{Mfset}
\SetKwData{Graph}{Graph}
\SetKwData{N}{n}
\SetKwData{Space}{ }
\SetKwData{Parent}{parent}
\SetKwData{Rank}{rank}
\SetKwData{Set}{Set}
\SetKwData{List}{List}
\SetKwData{Node}{Node}

\SetKwFunction{Max}{max}
\SetKwFunction{Insert}{insert}
\SetKwFunction{ListCos}{List}
\SetKwFunction{Head}{head}
\SetKwFunction{Len}{len}
\SetKwFunction{SetCos}{Set}


\SetKwProg{Fn}{}{}

\SetKwFunction{ProCamMin}{prototipoCamminiMinimi}
\SetKwFunction{}{}
\SetKwFunction{}{}
\SetKwFunction{}{}
\SetKwFunction{}{}
\SetKwFunction{}{}

\SetKw{Matrix}{matrix}

\caption{(\protect\Int[], \protect\Int[])  \protect\ProCamMin{\protect\Graph G, \protect\Node s}}
\comment{Inizializza $T$ ad una foresta di copertura composta da nodi isolati}
\comment{Inizializza $d$ con sovrastima della distanza: $d[s] = 0,\ d[x]=+\infty$}
\While{$\exists(u,\ v):d[u]\ +\ G.w(u,\ v)\ <\ d[v]$}{
	d[v] = d[u] + w(u, v)\;
	\comment{Sostituisci il padre di $v$ in $T$ con $u$}
}
\Return (T, d)\;
\end{algorithm}

Se al termine dell'esecuzione qualche nodo mantiene una distanza infinita non \`e raggiungibile.
\subsubsection{Algoritmo generico}
\begin{algorithm}[H]
\DontPrintSemicolon
\SetKwComment{comment}{$\%$}{}

\SetKw{Int}{int}
\SetKw{Bool}{boolean}
\SetKw{New}{new}
\SetKw{True}{true}
\SetKw{False}{false}
\SetKw{Not}{not}
\SetKw{And}{and}
\SetKw{Or}{or}
\SetKw{Down}{down}
\SetKw{To}{to}
\SetKw{New}{new}
\SetKw{Return}{return}
\SetKw{Nil}{nil}
\SetKw{Print}{print}
\SetKw{Void}{void}

\SetKwData{Item}{Item}
\SetKwData{Mfset}{Mfset}
\SetKwData{Graph}{Graph}
\SetKwData{N}{n}
\SetKwData{Space}{ }
\SetKwData{Parent}{parent}
\SetKwData{Rank}{rank}
\SetKwData{Set}{Set}
\SetKwData{List}{List}
\SetKwData{Node}{Node}

\SetKwFunction{Max}{max}
\SetKwFunction{Insert}{insert}
\SetKwFunction{ListCos}{List}
\SetKwFunction{Head}{head}
\SetKwFunction{Len}{len}
\SetKwFunction{SetCos}{Set}


\SetKwProg{Fn}{}{}

\SetKwFunction{ProCamMin}{shortestPath}
\SetKwFunction{}{}
\SetKwFunction{}{}
\SetKwFunction{}{}
\SetKwFunction{}{}
\SetKwFunction{}{}

\SetKw{Matrix}{matrix}

\caption{(\protect\Int[], \protect\Int[])  \protect\ProCamMin{\protect\Graph G, \protect\Node s}}
\Int[] d = \New \Int$[1\dots G.n]$\comment{$d[u]$ \`e la distanza da $s$ a $u$}
\Int[] T = \New \Int$[1\dots G.n]$\comment{$T[u]$ \`e il padre di $u$ nell'albero $T$}
\Bool[] b = \New \Bool$[1\dots G.n]$\comment{$b[u]$ \`e \textbf{true} se $u\in S$}
\ForEach{$u\in\ G.V()\ -\ \{s\}$}{
	T[u] = \Nil\;
	d[u] = $+\infty$\;
	b[u] = \False\;
}
T[s] = \Nil\;
d[s] = 0\;
b[s] = \True\;
$[\dots]$\;
\end{algorithm}


\begin{algorithm}[H]
\DontPrintSemicolon
\SetKwComment{comment}{$\%$}{}

\SetKw{Int}{int}
\SetKw{Bool}{boolean}
\SetKw{New}{new}
\SetKw{True}{true}
\SetKw{False}{false}
\SetKw{Not}{not}
\SetKw{And}{and}
\SetKw{Or}{or}
\SetKw{Down}{down}
\SetKw{To}{to}
\SetKw{New}{new}
\SetKw{Return}{return}
\SetKw{Nil}{nil}
\SetKw{Print}{print}
\SetKw{Void}{void}

\SetKwData{Item}{Item}
\SetKwData{Mfset}{Mfset}
\SetKwData{Graph}{Graph}
\SetKwData{N}{n}
\SetKwData{Space}{ }
\SetKwData{Parent}{parent}
\SetKwData{Rank}{rank}
\SetKwData{Set}{Set}
\SetKwData{List}{List}
\SetKwData{Node}{Node}
\SetKwData{DataStructure}{DataStructure}

\SetKwFunction{Max}{max}
\SetKwFunction{Insert}{insert}
\SetKwFunction{ListCos}{List}
\SetKwFunction{Head}{head}
\SetKwFunction{Len}{len}
\SetKwFunction{SetCos}{Set}
\SetKwFunction{DataStructureCos}{Datastructure}
\SetKwFunction{Add}{add}
\SetKwFunction{Adj}{adj}
\SetKwFunction{IsEmpty}{isEmpty}
\SetKwFunction{Extract}{extract}

\SetKwProg{Fn}{}{}

\SetKwFunction{ProCamMin}{shortestPath}
\SetKwFunction{}{}
\SetKwFunction{}{}
\SetKwFunction{}{}
\SetKwFunction{}{}
\SetKwFunction{}{}

\SetKw{Matrix}{matrix}

\caption{(\protect\Int[], \protect\Int[])  \protect\ProCamMin{\protect\Graph G, \protect\Node s}}
\DataStructure S = \DataStructureCos{}\;
S.\Add{s}\;
\While{\Not S.\IsEmpty{}}{
	\Int u = S.\Extract{}\;
	b[u] = \False\;
	\ForEach{v $\in$ G.\Adj{u}}{
		\If{d[u]+G.w(u, v) $<$ d[v]}{
			\uIf{\Not b[v]}{
				S.\Add{v}\;
				b[v] = \True\;
			}
			\Else{
				\comment{Azione da svolgere nel caso $v$ sia gi\`a presente in $S$}
			}
			T[v] = u\;
			d[v] = d[u] + G.w(u, v)\;
		}
	}
}
\Return (T, d)\;
\end{algorithm}

\section{Algoritmo di Dijkstra}
Nella versione originale viene utilizzato per trovare la distanza minima fra due nodi utilizzando il concetto di code di priorit\`a. Funziona unicamente con pesi positivi e viene 
utilizzato in protocolli di rete.
\subsection{Funzionamento}
\subsubsection{Inizializzazione}
Viene creato un vettore di dimensione $n$ in cui l'indice $u$ rappresenta il nodo $u$-esimo. Le priorit\`a vengono inizializzate a $+\infty$ e quella di $s$ uguale a $0$. Questa 
operazione ha costo computazionale $O(n)$.
\subsubsection{Estrazione minimo}
Si ricerca il minimo all'interno del vettore, una volta trovato si cancella la sua priorit\`a, con costo computazionale $O(n)$. 
\subsubsection{Inserimento in coda}
Si registra la priorit\`a nella posizione $v$-esima con costo $O(1)$.
\subsubsection{Aggiornamento priorit\`a}
Si aggiorna la priorit\`a nbella posizione $v$-esima con costo computazionale $O(1)$.
\subsection{Correttezza per pesi positivi}
Ogni nodo viene estratto una e una sola volta e al momento dell'estrazione la sua distanza \`e minima. La dimostrazione si fa per induzione sul numero $k$ di nodi estratti.
\subparagraph{Caso base}
\`E vero in quanto $d[s]=0$ e non ci sono lunghezze negative.
\subparagraph{Passo induttivo}
Si consideri per ipotesi induttiva che sia vero per i primi $k-1$ nodi. Quando viene estratto il $k$-esimo nodo $u$ la sua distanza $d[u]$ dipende dai $k-1$ nodi gi\`a estratti e non 
pu\`o dipendere dai nodi ancora da estrarre in quanto hanno distanza $\ge d[u]$, pertanto $d[u]$ \`e minimo e $u$ non verr\`a pi\`u re-inserito in quanto non ci sono distanze negative.
\subsubsection{Implementazione}
\begin{algorithm}[H]
\DontPrintSemicolon
\SetKwComment{comment}{$\%$}{}

\SetKw{Int}{int}
\SetKw{Bool}{boolean}
\SetKw{New}{new}
\SetKw{True}{true}
\SetKw{False}{false}
\SetKw{Not}{not}
\SetKw{And}{and}
\SetKw{Or}{or}
\SetKw{Down}{down}
\SetKw{To}{to}
\SetKw{New}{new}
\SetKw{Return}{return}
\SetKw{Nil}{nil}
\SetKw{Print}{print}
\SetKw{Void}{void}

\SetKwData{Item}{Item}
\SetKwData{Mfset}{Mfset}
\SetKwData{Graph}{Graph}
\SetKwData{N}{n}
\SetKwData{Space}{ }
\SetKwData{Parent}{parent}
\SetKwData{Rank}{rank}
\SetKwData{Set}{Set}
\SetKwData{List}{List}
\SetKwData{Node}{Node}
\SetKwData{DataStructure}{DataStructure}
\SetKwData{PriorityQueue}{PriorityQueue}

\SetKwFunction{Max}{max}
\SetKwFunction{Insert}{insert}
\SetKwFunction{ListCos}{List}
\SetKwFunction{Head}{head}
\SetKwFunction{Len}{len}
\SetKwFunction{SetCos}{Set}
\SetKwFunction{DataStructureCos}{Datastructure}
\SetKwFunction{PriorityQueueCos}{PriorityQueue}
\SetKwFunction{DeleteMin}{deleteMin}
\SetKwFunction{Decrease}{decrease}
\SetKwFunction{Add}{add}
\SetKwFunction{Adj}{adj}
\SetKwFunction{IsEmpty}{isEmpty}
\SetKwFunction{Extract}{extract}

\SetKwProg{Fn}{}{}

\SetKwFunction{ProCamMin}{shortestPath}
\SetKwFunction{}{}
\SetKwFunction{}{}
\SetKwFunction{}{}
\SetKwFunction{}{}
\SetKwFunction{}{}

\SetKw{Matrix}{matrix}

\caption{(\protect\Int[], \protect\Int[])  \protect\ProCamMin{\protect\Graph G, \protect\Node s}}
\PriorityQueue Q = \PriorityQueueCos{}\;
Q.\Insert{s, 0}\;
\While{\Not Q.\IsEmpty{}}{
	\Int u = Q.\DeleteMin{}\;
	b[u] = \False\;
	\ForEach{v $\in$ G.\Adj{u}}{
		\If{d[u]+G.w(u, v) $<$ d[v]}{
			\uIf{\Not b[v]}{
				Q.\Insert{v, d[u]+G.w(u, v)}\;
				b[v] = \True\;
			}
			\Else{
				Q.\Decrease{v, D[u]+G.w(u, v)}\;
			}
			T[v] = u\;
			d[v] = d[u] + G.w(u, v)\;
		}
	}
}
\Return (T, d)\;
\end{algorithm}

\paragraph{Costo computazionale con coda con priorit\`a basata su vettore}
\begin{itemize}
	\item Inizializzazione: costo $O(n)$ con una ripetizione.
	\item Estrazione minimo: costo $O(n)$ con $O(n)$ ripetizioni.
	\item Inserimento in coda: costo $O(1)$ con $O(n)$ ripetizioni.
	\item Aggiornamento priorit\`a: costo $O(1)$ con $O(m)$ ripetizioni.
\end{itemize}
Per un costo totale di $O(n^2)$.
\paragraph{Costo computazionale con coda con priorit\`a basata su heap binario}
\begin{itemize}
	\item Inizializzazione: costo $O(n)$ con una ripetizione.
	\item Estrazione minimo: costo $O(\log n)$ con $O(n)$ ripetizioni.
	\item Inserimento in coda: costo $O(\log n)$ con $O(n)$ ripetizioni.
	\item Aggiornamento priorit\`a: costo $O(\log n)$ con $O(m)$ ripetizioni.
\end{itemize}
Per un costo totale di $O(m\log n)$.
\paragraph{Costo computazionale con coda con priorit\`a basata su heap binario}
\begin{itemize}
	\item Inizializzazione: costo $O(n)$ con una ripetizione.
	\item Estrazione minimo: costo $O(\log n)$ con $O(n)$ ripetizioni.
	\item Inserimento in coda: costo $O(\log n)$ con $O(n)$ ripetizioni.
	\item Aggiornamento priorit\`a: costo ammortizzato $O(1)$ con $O(m)$ ripetizioni.
\end{itemize}
Per un costo totale di $O(m+n\log n)$.
\section{Coda Bellman-Ford, Moore}
Questo algoritmo \`e computazionalmente pi\`u pesante di Dijkstra, ma funziona anche con archi di peso negativo.
\subsubsection{Inizializzazione}
Viene creata una coda di dimensione $n$ con costo computazionale $O(n)$.
\subsubsection{Estrazione}
Viene estratto il prossimo elemento della coda con costo computazionale $O(1)$.
\subsubsection{Inserimento in coda}
Si inserici l'indice $v$ in coda con costo computazionale $O(1)$. Non si deve considerare il caso in cui $v$ sia gi\`a presente in $S$.
\subsection{Correttezza}
Si definisce una passata:
\begin{itemize}
	\item Per $k=0$ la zeresima passata consiste nell'estrazione del nodo $s$ dalla coda $S$.
	\item Per $k>0$ la $k$-esima passata consiste nell'estrazione di tutti i nodi presenti in $S$ al termine della passata $k-1$-esima.
\end{itemize}
Al termine della passata $k$ i vettori $T$ e $d$ descrivono i cammini minimi di lunghezza al pi\`u $k$, al termine della passata $n-1$ descrivono i cammini minimi di lunghezza al pi\`u
$n-1$. 
\subsubsection{Implementazione}
\begin{algorithm}[H]
\DontPrintSemicolon
\SetKwComment{comment}{$\%$}{}

\SetKw{Int}{int}
\SetKw{Bool}{boolean}
\SetKw{New}{new}
\SetKw{True}{true}
\SetKw{False}{false}
\SetKw{Not}{not}
\SetKw{And}{and}
\SetKw{Or}{or}
\SetKw{Down}{down}
\SetKw{To}{to}
\SetKw{New}{new}
\SetKw{Return}{return}
\SetKw{Nil}{nil}
\SetKw{Print}{print}
\SetKw{Void}{void}

\SetKwData{Item}{Item}
\SetKwData{Mfset}{Mfset}
\SetKwData{Graph}{Graph}
\SetKwData{N}{n}
\SetKwData{Space}{ }
\SetKwData{Parent}{parent}
\SetKwData{Rank}{rank}
\SetKwData{Set}{Set}
\SetKwData{List}{List}
\SetKwData{Node}{Node}
\SetKwData{DataStructure}{DataStructure}
\SetKwData{PriorityQueue}{PriorityQueue}
\SetKwData{Queue}{Queue}

\SetKwFunction{Max}{max}
\SetKwFunction{Insert}{insert}
\SetKwFunction{ListCos}{List}
\SetKwFunction{Head}{head}
\SetKwFunction{Len}{len}
\SetKwFunction{SetCos}{Set}
\SetKwFunction{DataStructureCos}{Datastructure}
\SetKwFunction{PriorityQueueCos}{PriorityQueue}
\SetKwFunction{DeleteMin}{deleteMin}
\SetKwFunction{Decrease}{decrease}
\SetKwFunction{Add}{add}
\SetKwFunction{Adj}{adj}
\SetKwFunction{IsEmpty}{isEmpty}
\SetKwFunction{Extract}{extract}
\SetKwFunction{Dequeue}{dequeue}
\SetKwFunction{Enqueue}{enqueue}

\SetKwProg{Fn}{}{}

\SetKwFunction{ProCamMin}{shortestPath}
\SetKwFunction{}{}
\SetKwFunction{}{}
\SetKwFunction{}{}
\SetKwFunction{}{}
\SetKwFunction{}{}

\SetKw{Matrix}{matrix}

\caption{(\protect\Int[], \protect\Int[])  \protect\ProCamMin{\protect\Graph G, \protect\Node s}}
\Queue Q = \QueueCos{}\;
Q.\Enqueue{s}\;
\While{\Not Q.\IsEmpty{}}{
	\Int u = Q.\Dequeue{}\;
	b[u] = \False\;
	\ForEach{v $\in$ G.\Adj{u}}{
		\If{d[u]+G.w(u, v) $<$ d[v]}{
			\uIf{\Not b[v]}{
				Q.\Enqueue{v}\;
				b[v] = \True\;
			}
			\Else{
				Q.\Decrease{v, D[u]+G.w(u, v)}\;
			}
			T[v] = u\;
			d[v] = d[u] + G.w(u, v)\;
		}
	}
}
\Return (T, d)\;
\end{algorithm}

\subsubsection{Complessit\`a}
\begin{itemize}
	\item Inizializzazione: costo $O(1)$ ripetuto una volta.
	\item Estrazione: costo $O(1)$ ripetuto $O(n^2)$ volte.
	\item Inserimento in coda: costo $O(1)$ ripetuto $O(nm)$ volte.
\end{itemize}
Il costo totale \`e pertanto $O(mn)$. Ogni nodo pu\`o essere inserito ed estratto al massimo $n-1$ volte.
\section{Cammini minimi su DAG}
I cammini minimi in un DAG sono sempre ben definiti, anche in presenza di pesi negativi in quanto non esistono cicli negativi. \`E possibile rilassare gli archi in ordine topologico una
volta sola e non essendoci cicli non \`e possibile tornare su un nodo gi\`a visitato e abbassare il valore del suo campo $d$. L'algoritmo utilizza pertanto l'ordinamento topologico.\\
\begin{algorithm}[H]
\DontPrintSemicolon
\SetKwComment{comment}{$\%$}{}

\SetKw{Int}{int}
\SetKw{Bool}{boolean}
\SetKw{New}{new}
\SetKw{True}{true}
\SetKw{False}{false}
\SetKw{Not}{not}
\SetKw{And}{and}
\SetKw{Or}{or}
\SetKw{Down}{down}
\SetKw{To}{to}
\SetKw{New}{new}
\SetKw{Return}{return}
\SetKw{Nil}{nil}
\SetKw{Print}{print}
\SetKw{Void}{void}

\SetKwData{Item}{Item}
\SetKwData{Mfset}{Mfset}
\SetKwData{Graph}{Graph}
\SetKwData{N}{n}
\SetKwData{Space}{ }
\SetKwData{Parent}{parent}
\SetKwData{Rank}{rank}
\SetKwData{Set}{Set}
\SetKwData{List}{List}
\SetKwData{Node}{Node}
\SetKwData{DataStructure}{DataStructure}
\SetKwData{PriorityQueue}{PriorityQueue}
\SetKwData{Queue}{Queue}
\SetKwData{Stack}{Stack}

\SetKwFunction{Max}{max}
\SetKwFunction{Insert}{insert}
\SetKwFunction{ListCos}{List}
\SetKwFunction{Head}{head}
\SetKwFunction{Len}{len}
\SetKwFunction{SetCos}{Set}
\SetKwFunction{DataStructureCos}{Datastructure}
\SetKwFunction{PriorityQueueCos}{PriorityQueue}
\SetKwFunction{DeleteMin}{deleteMin}
\SetKwFunction{Decrease}{decrease}
\SetKwFunction{Add}{add}
\SetKwFunction{Adj}{adj}
\SetKwFunction{IsEmpty}{isEmpty}
\SetKwFunction{Extract}{extract}
\SetKwFunction{Dequeue}{dequeue}
\SetKwFunction{Enqueue}{enqueue}
\SetKwFunction{TopSort}{topSort}
\SetKwFunction{Pop}{pop}

\SetKwProg{Fn}{}{}

\SetKwFunction{ProCamMin}{shortestPath}
\SetKwFunction{}{}
\SetKwFunction{}{}
\SetKwFunction{}{}
\SetKwFunction{}{}
\SetKwFunction{}{}

\SetKw{Matrix}{matrix}

\caption{(\protect\Int[], \protect\Int[])  \protect\ProCamMin{\protect\Graph G, \protect\Node s}}
\Int[] d = \New \Int$[1\dots G.n]$ \comment{$d[u]$ \`e la distanza da $s$ a $u$}
\Int[] T = \New \Int$[1\dots G.n]$ \comment{$T[u]$ \`e il padre di $u$ nell'albero $T$}
\ForEach{u $\in$ G.V() - $\{s\}$}{
	T[u] = \Nil\;
	d[u] = $+\infty$\;
}
T[s] = \Nil\;
d[s] = 0\;
\Stack S = \TopSort{G}\;
\While{\Not S.\IsEmpty{}}{
	u = S.\Pop{}\;
	\ForEach{v $\in$ G.\Adj{u}}{
		\If{d[u] + G.w(u, v) $<$ d[v]}{
			T[v] = u\;
			d[v] = d[u] + G.w(u, v)\;
		}
	}
}
\Return (T, d)\;
\end{algorithm}

\section{Conclusioni}
\begin{center}
\begin{tabular}{|c|c|c|}
	\hline
	Dijkstra & $O(n^2)$ & Pesi positivi, grafi densi\\
	\hline
	Johnson & $O(m\log n)$ & Pesi positivi, grafi sparsi\\
	\hline
	Fredman-Tarjan & $O(m+n\log n)$ & \makecell{Pesi positivi, grafi densi \\ dimensioni molto grandi}\\
	\hline
	Bellman-Ford & $O(mn)$ & Pesi negativi\\
	\hline
		     & $O(m+n)$ & DAG\\
	\hline
	DFS & $O(m+n)$ & Senza pesi\\
	\hline
\end{tabular}
\end{center}
\section{Cammini minimi, sorgente multipla}
\begin{center}
\begin{tabular}{|c|c|c|}
	\hline
	\textbf{Input} & \textbf{Complessit\`a} & \textbf{Approccio}\\
	\hline
	\makecell{Pesi positivi, \\ grafo denso} & $O(n\cdot n^2)$ & \makecell{Applicazione ripetuta dell'algoritmo \\ di Dijkstra}\\
	\hline
	\makecell{Pesi positivi, \\ grafo sparso}  & $O(n\cdot (m\log n))$ & \makecell{Applicazione ripetuta dell'algoritmo \\ di Johnson}\\
	\hline
	Pesi negativi & $O(n\cdot mn)$ & \makecell{Applicazione ripetuta di \\ Bellman-Ford, sconsigliata}\\
	\hline
	\makecell{Pesi negativi, \\ grafo denso} & $O(n^3)$ & Algoritmo di Floyd e Warshall\\
	\hline
	\makecell{Pesi negativi, \\ grafo sparso} & $O(nm\log n)$ & \makecell{Algoritmo di Johnson per sorgente \\ multipla}\\
	\hline
\end{tabular}
\end{center}
\section{Floyd-Warshall}
\subsection{Cammini minimi $k$-vincolati}
Sia $k$ un valore in $\{0, \dots, n\}$. Si dice che un cammino $p^k_{xy}$ \`e un cammino $k$-vincolato fra $x$ e $y$ se esso ha il costo minimo fra tutti i cammini fra $x$ e $y$ che non 
passano per nessun vertice in $v_{k+1}, \dots, v_n$ ($x$ e $y$ sono esclusi dal vincolo). Si assuma che esista un ordinamento tra i nodi del grafo.
\subsection{Distanza $k$-vincolata}
Si denota con $d^k[x][y]$ il costo totale del cammino minimo $k$-vincolato fra $x$ e $y$ se esiste.
$$ d^k[x][y] = \begin{cases}
	w(p^k_{xy}) & se\ esiste\ p^k_{xy}\\
	+\infty & altrimenti\\
\end{cases}$$
Si pu\`o definire ricorsivamente come:
$$ d^k[x][y] = \begin{cases}
	w(x, y) & k = 0\\
	\min(d^{k-1}[x][y], d^{k-1}[x][k]+d^{k-1}[k][y]) & k>0\\
\end{cases}$$
\subsubsection{Matrice dei padri}
Oltre a definire $d$ si definisce la matrice $T$, dove $T[x][y]$ rappresenta il predecessore di $y$ nel cammino pi\`u breve da $x$ a $y$.\\
\begin{algorithm}[H]
\DontPrintSemicolon
\SetKwComment{comment}{$\%$}{}

\SetKw{Int}{int}
\SetKw{Bool}{boolean}
\SetKw{New}{new}
\SetKw{True}{true}
\SetKw{False}{false}
\SetKw{Not}{not}
\SetKw{And}{and}
\SetKw{Or}{or}
\SetKw{Down}{down}
\SetKw{To}{to}
\SetKw{New}{new}
\SetKw{Return}{return}
\SetKw{Nil}{nil}
\SetKw{Print}{print}
\SetKw{Void}{void}

\SetKwData{Item}{Item}
\SetKwData{Mfset}{Mfset}
\SetKwData{Graph}{Graph}
\SetKwData{N}{n}
\SetKwData{Space}{ }
\SetKwData{Parent}{parent}
\SetKwData{Rank}{rank}
\SetKwData{Set}{Set}
\SetKwData{List}{List}
\SetKwData{Node}{Node}
\SetKwData{DataStructure}{DataStructure}
\SetKwData{PriorityQueue}{PriorityQueue}
\SetKwData{Queue}{Queue}
\SetKwData{Stack}{Stack}

\SetKwFunction{Max}{max}
\SetKwFunction{Insert}{insert}
\SetKwFunction{ListCos}{List}
\SetKwFunction{Head}{head}
\SetKwFunction{Len}{len}
\SetKwFunction{SetCos}{Set}
\SetKwFunction{DataStructureCos}{Datastructure}
\SetKwFunction{PriorityQueueCos}{PriorityQueue}
\SetKwFunction{DeleteMin}{deleteMin}
\SetKwFunction{Decrease}{decrease}
\SetKwFunction{Add}{add}
\SetKwFunction{Adj}{adj}
\SetKwFunction{IsEmpty}{isEmpty}
\SetKwFunction{Extract}{extract}
\SetKwFunction{Dequeue}{dequeue}
\SetKwFunction{Enqueue}{enqueue}
\SetKwFunction{TopSort}{topSort}
\SetKwFunction{Pop}{pop}

\SetKwProg{Fn}{}{}

\SetKwFunction{ProCamMin}{floydWarshall}
\SetKwFunction{}{}
\SetKwFunction{}{}
\SetKwFunction{}{}
\SetKwFunction{}{}
\SetKwFunction{}{}

\SetKw{Matrix}{matrix}

\caption{(\protect\Int[], \protect\Int[])  \protect\ProCamMin{\protect\Graph G}}
\Int[] d = \New \Int$[1\dots G.n]$\;
\Int[] T = \New \Int$[1\dots G.n]$\;
\ForEach{u, v $\in$ G.V() - $\{s\}$}{
	T[u][v] = \Nil\;
	d[u][v] = $+\infty$\;
}
T[s] = \Nil\;
d[s] = 0\;
\ForEach{u $\in$ G.V()}{
	\ForEach{v $\in$ G.\Adj{u}}{
		d[u][v] = G.w(u, v)\;
		T[u][v] = u\;
	}
}
\For{\Int k = 1 \To G.n}{
	\ForEach{u $\in$ G.V()}{
		\ForEach{v $\in$ G.V()}{
			\If{d[u][k] + d[k][v] $<$ d[u][v]}{
				d[u][v] = d[u][k] + d[k][u]\;
				T[u][v] = T[k][v]\;
			}
		}
	}
}
\Return (T, d)\;
\end{algorithm}

\section{Chiusura transitiva (algoritmo di Warshall)}
La chiusura transitiva $G^* = (V, E^*)$ di un grafo $G = (V, E)$ \`e il grafo orientato tale che $(u, v)\in E^*$ se e solo se esiste un cammino da $u$ a $v$ in $G$. Supponendo di avere il
grafo $G$ rappresentato da una matrice di adiacenza $M$, la matrice $M^n$ rappresenta la matrice di adiacenza di $G^*$.
\subsubsection{Formulazione ricorsiva}
$$M^k[x][y] = \begin{cases}
	M[x][y] & k = 0\\
	M^{k-1}[x][y]\lor (M^{k-1}[x][k]\land M^{k-1}[k][y] & k>0\\
\end{cases}$$
