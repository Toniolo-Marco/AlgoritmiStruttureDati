\chapter{Alberi}
\section{Alberi radicati}
Un albero radicato pu\`o essere definito in due modi.
\subsection{Definizioni}
\subsubsection{Definizione chiusa}
Un albero radicato consiste in una serie di nodi e un insieme di archi orientati che connettono coppie di nodi con le seguenti propriet\`a:
\begin{itemize}
\item Un nodo dell'albero \`e designato come radice.
\item Ogni nodo $n$, a parte la radice ha esattamente un arco entrante.
\item Esiste un cammino unico dalla radice ad ogni nodo.
\item L'albero \`e connesso.
\end{itemize}
\subsubsection{Definizione ricorsiva}
Un albero radicato \`e dato da:
\begin{itemize}
\item Un insieme vuoto.
\item Un nodo radice e zero o pi\`u sottoalberi ognuno dei quali \`e un albero, la radice \`e connessa alla radice di ogni sottoalbero con un arco orientato.
\end{itemize}
\subsection{Terminologia}
\begin{itemize}
\item Il nodo senza archi entranti \`e detto radice (root).
\item Per generare i sottoalberi (subtrees) da un albero si elimina la radice e tutti i suoi archi uscenti.
\item Nodi con lo stesso genitore sono detti fratelli (siblings).
\item Considerando un arco, il nodo da cui parte \`e detto genitore (parent) del nodo in cui arriva, detto figlio (child).
\item I nodi senza archi uscenti sono detti foglie (leaves).
\item I nodi n\`e foglie n\`e radice sono detti interni (internal nodes).
\item La lunghezza del cammino semplice dalla radice ad un nodo, misurato nel numero di archi \`e detto profondit\`a (depth) del nodo.
\item I nodi alla stessa profondit\`a formano un insieme chiamato livello (level).
\item La profondit\`a massima dell'albero si dice altezza (height).
\end{itemize}
\section{Visite di alberi}
La visita di un albero o ricerca \`e una strategia per visitare tutti i nodi di un albero. Il costo computazionale di una visita su un albero su $n$ nodi
\`e $\Theta(n)$ in quanto ogni nodo viene visitato un'unica volta.
\subsection{Visita in profondit\`a}
La visita in profondit\`a o depth-first-search (dfs) si compie visitando ricorsivamente tutti i sottoalberi dell'albero, richiede uno stack ed esiste in tre 
varianti: pre, in e post order.
\subsection{Visita in ampiezza}
La visita in ampiezza o breadth-fisrt-search (bfs) visita completamente ogni livello prima di passare al successivo partendo dalla radice. Richiede una 
queue.
\section{Albero binario}
Un albero binario \`e un albero radicato in cui ogni nodo ha al massimo due figli, indicati con figlio destro e sinistro. Due alberi binari differiscono 
anche se uno stesso nodo \`e designato come figlio sinistro invece che destro o viceversa.
\subsection{Specifica}
\begin{algorithm}[H]
\DontPrintSemicolon
\SetKw{Boolean}{boolean}
\SetKw{Nil}{nil}
\SetKwData{Item}{Item}
\SetKwData{Tree}{Tree}
\SetKwComment{comment}{$\%$}{}
\SetKwFunction{TreeCos}{Tree}
\SetKwFunction{Read}{read}
\SetKwFunction{Write}{write}
\SetKwFunction{Parent}{Parent}
\SetKwFunction{Left}{left}
\SetKwFunction{Right}{right}
\SetKwFunction{InsertLeft}{insertLeft}
\SetKwFunction{InsertRight}{insertRight}
\SetKwFunction{DeleteLeft}{deleteLeft}
\SetKwFunction{DeleteRight}{deleteRight}

\caption{\protect\Tree}

\comment{Costruisce un nuovo nodo contenente $v$ senza figli o genitori}
\TreeCos{\Item v}\;
\comment{Legge il valore memorizzato nel nodo}
\Item \Read{}\;
\comment{Modifica il valore memorizzato nel nodo}
\Write{\Item v}\;
\comment{Restituisce il padre o \Nil se \`e il nodo radice}
\Tree \Parent{}\;
\comment{Restituisce il figlio sinistro (destro) di questo nodo o \Nil se assente}
\Tree \Left{}\;
\Tree \Right{}\;
\comment{Inserisce il sottoalbero radicato in $t$ come figlio sinistro (destro) di questo nodo}
\InsertLeft{\Tree t}\;
\InsertRight{\Tree t}\;
\comment{Distrugge ricorsivamente il figlio sinistro (destro) di questo nodo}
\DeleteLeft{}\;
\DeleteRight{}\;
\end{algorithm}
\subsection{Memorizzazione e implementazione}
Ogni nodo deve memorizzare oltre al proprio valore la reference al nodo padre (parent), la reference ai figli sinistro (left) e destro (right).
\subsubsection{Implementazione}
\begin{algorithm}[H]
\DontPrintSemicolon
\SetKwComment{comment}{$\%$}{}
\SetKwData{Item}{Item}
\SetKwData{Tree}{Tree}
\SetKwData{Binary}{Binary tree}
\SetKw{Boolean}{boolean}
\SetKw{Int}{int}
\SetKw{This}{this}
\SetKw{Return}{return}
\SetKw{Prec}{precondition}
\SetKw{New}{new}
\SetKw{Space}{ }
\SetKw{Nil}{nil}
\SetKwProg{Fn}{}{}{}
\SetKwFunction{TreeCos}{Tree}
\SetKwFunction{InsertLeft}{insertLeft}
\SetKwFunction{InsertRight}{insertRight}
\SetKwFunction{DeleteLeft}{deleteLeft}
\SetKwFunction{DeleteRight}{deleteRight}

\caption{\protect\Binary}

\begin{multicols}{2}
\Fn{\TreeCos{\Item v}}{
	\Tree t = \New \Tree\;
	t.parent = \Nil\;
	t.left = t.right = \Nil\;
	t.value = v\;
	\Return t\;
}
\Fn{\InsertLeft{\Tree T}}{
	\If{left == \Nil}{
    	T.parent = \This\;
    	left = T\;
  }
}
\Fn{\InsertRight{\Tree T}}{
	\If{right == \Nil}{
    	T.parent = \This\;
    	right = T\;
  }
}

\Fn{\DeleteLeft{}}{
	\If{left $\neq$ \Nil}{
    	left.\DeleteLeft{}\;
    	left.\DeleteRight{}\;
    	left = \Nil\;
  }
}

\Fn{\DeleteRight{}}{
	\If{right $\neq$ \Nil}{
    	right.\DeleteLeft{}\;
    	right.\DeleteRight{}\;
    	right = \Nil\;
  }
}
\end{multicols}
\end{algorithm}
\subsection{Visita in profondit\`a}
\begin{algorithm}[H]
\DontPrintSemicolon
\SetKwComment{comment}{$\%$}{}
\SetKw{Nil}{nil}
\SetKw{Print}{print}
\SetKwData{Tree}{Tree}
\SetKwFunction{Dfs}{dfs}
\SetKwFunction{Left}{left}
\SetKwFunction{Right}{right}
\SetKwFunction{Dfs}{dfs}
\caption{\protect\Dfs{\protect\Tree t}}

\If{t $\neq$ \Nil}{
\comment{visita in pre-ordine}
\Print t\;
\Dfs{t.\Left}\;
\comment{visita in in-ordine}
\Print t\;
\Dfs{t.\Right}\;
\comment{visita in post-ordine}
\Print t\;
}
\end{algorithm}
\section{Alberi generici}
\subsection{Specifica}
\begin{algorithm}[H]
\DontPrintSemicolon
\SetKw{Boolean}{boolean}
\SetKw{Nil}{nil}
\SetKwData{Item}{Item}
\SetKwData{Tree}{Tree}
\SetKwComment{comment}{$\%$}{}
\SetKwFunction{TreeCos}{Tree}
\SetKwFunction{Read}{read}
\SetKwFunction{Write}{write}
\SetKwFunction{Parent}{Parent}
\SetKwFunction{Left}{leftmostchild}
\SetKwFunction{Right}{rightSibling}
\SetKwFunction{InsertChild}{insertChild}
\SetKwFunction{InsertSibling}{insertSibling}
\SetKwFunction{DeleteLeft}{deleteChild}
\SetKwFunction{DeleteRight}{deleteSibling}

\caption{\protect\Tree}

\comment{Costruisce un nuovo nodo contenente $v$ senza figli o genitori}
\TreeCos{\Item v}\;
\comment{Legge il valore memorizzato nel nodo}
\Item \Read{}\;
\comment{Modifica il valore memorizzato nel nodo}
\Write{\Item v}\;
\comment{Restituisce il padre o \Nil se \`e il nodo radice}
\Tree \Parent{}\;
\comment{Restituisce il primo figlio o \Nil se \`e un nodo foglia}
\Tree \Left{}\;
\comment{Restituisce il prossimo fratello o \Nil se assente}
\Tree \Right{}\;
\comment{Inserisce il sottoalbero $t$ come primo figlio di questo nodo}
\InsertChild{\Tree t}\;
\comment{Inserisce il sottoalbero $t$ come prossimo fratello di questo nodo}
\InsertSibling{\Tree t}\;
\comment{Distrugge l'albero radicato nel primo figlio}
\DeleteLeft{}\;
\comment{Distrugge l'albero radicato nel prossimo fratello}
\DeleteRight{}\;
\end{algorithm}
\subsubsection{Depth-first search}
\begin{algorithm}[H]
\DontPrintSemicolon
\SetKwComment{comment}{$\%$}{}
\SetKw{Nil}{nil}
\SetKw{Print}{print}
\SetKwData{Tree}{Tree}
\SetKwFunction{Dfs}{dfs}
\SetKwFunction{Left}{leftMostChild}
\SetKwFunction{Right}{rightSibling}
\SetKwFunction{Dfs}{dfs}

\caption{\protect\Dfs{\protect\Tree t}}

\If{t $\neq$ \Nil}{
	\comment{visita in pre-ordine}
	\Print t\;
	\Tree u = t.\Left{}\;
	\While{u $\neq$ \Nil}{
		\Dfs{u}\;
		u = u.\Right{}\;
	}
	\comment{visita in post-ordine}
	\Print t\;
}
\end{algorithm}
\subsubsection{Breadth-first search}
\begin{algorithm}[H]
\DontPrintSemicolon
\SetKwComment{comment}{$\%$}{}
\SetKw{Nil}{nil}
\SetKw{Not}{not}
\SetKw{Print}{print}
\SetKwData{Tree}{Tree}
\SetKwData{Queue}{Queue}
\SetKwFunction{Bfs}{bfs}
\SetKwFunction{QueueCos}{Queue}
\SetKwFunction{Enqueue}{enqueue}
\SetKwFunction{Dequeue}{dequeue}
\SetKwFunction{Left}{leftMostChild}
\SetKwFunction{Right}{rightSibling}
\SetKwFunction{IsEmpty}{isEmpty}
\SetKwFunction{Bfs}{bfs}

\caption{\protect\Bfs(\protect\Tree t)}

\Queue Q = \QueueCos{}\;
Q.\Enqueue{t}\;
\While{\Not Q.\IsEmpty{}}{
	\Tree u = Q.\Dequeue{}\;
	\comment{Visita per livelli nodo $u$}
	\Print u\;
	u = u.\Left{}\;
	\While{u $\neq$ \Nil}{
		Q.\Enqueue{u}\;
		u = u.\Right{}\;	
	}
}
\end{algorithm}
\subsection{Memorizzazione}
Esistono vari modi per salvare un albero, scelti in base al numero massimo e medio dei figli presenti.
\subsubsection{Vettore dei figli}
Nel nodo viene memorizzato il genitore e un vettore contenente i figli, che a seconda del loro numero pu\`o portare a uno spreco dello spazio.
\newpage
\subsubsection{Primo figlio, prossimo fratello}
\begin{algorithm}[H]
\DontPrintSemicolon
\SetKwComment{comment}{$\%$}{}
\SetKwData{Item}{Item}
\SetKwData{Tree}{Tree}
\SetKw{Boolean}{boolean}
\SetKw{Int}{int}
\SetKw{This}{this}
\SetKw{Return}{return}
\SetKw{Prec}{precondition}
\SetKw{New}{new}
\SetKw{Space}{ }
\SetKw{Nil}{nil}
\SetKw{Self}{self}
\SetKwProg{Fn}{}{}{}
\SetKwFunction{TreeCos}{Tree}
\SetKwFunction{InsertChild}{insertChild}
\SetKwFunction{InsertSibling}{insertSibling}
\SetKwFunction{DeleteChild}{deleteChild}
\SetKwFunction{DeleteSibling}{deleteSibling}
\SetKwFunction{Delete}{delete}
\SetKwFunction{RightSibling}{rightSibling}
\SetKwFunction{LeftMostChild}{leftMostChild}

\caption{\protect\Tree}

\begin{multicols}{2}
\Tree parent    \comment{Reference al padre}
\Tree child		\comment{Reference al primo figlio}
\Tree sibling 	\comment{Reference al prossimo fratello}
\Item value		\comment{Valore memorizzato nel nodo}





\Fn{\TreeCos{\Item v}}{
	\Tree t = \New \Tree\;
	t.value = v\;
	t.parent = t.child = t.sibling = \Nil\;
	\Return t\;
}

\Fn{\InsertChild{\Tree t}}{
	t.parent = \Self\;
	t.sibling = child\;
    child = t\;
}

\Fn{\InsertSibling{\Tree t}}{
	t.parent = parent\;
	t.sibling = sibling\;
	sibling = t\;
}

\Fn{\DeleteChild{}}{
	\Tree newChild = child.\RightSibling{}
	\Delete{child}\;
	child = newChild\;
}

\Fn{\DeleteSibling{}}{
	\Tree newBrother = sibling.\RightSibling{}\;
	\Delete{sibling}\;
	sibling = newBrother\;
	
}
\Fn{\Delete{\Tree t}}{
	\Tree u = t.\LeftMostChild{}\;
	\While{u $\neq$ \Nil}{
		\Tree next = u.\RightSibling{}\;
		\Delete{u}\;
		u = next\;	
	}
}
\end{multicols}
\end{algorithm}
\subsubsection{Vettore dei padri}
L'albero \`e rappresentato da un vettore i cui elementi contengono il valore associato al nodo e l'indice della posizione del padre nel vettore.
