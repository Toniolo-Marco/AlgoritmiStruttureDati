\chapter{Algoritmi di ordinamento}
Si considerino gli algoritmi di ordinamento:
\begin{multicols}{2}
	\begin{itemize}
		\item \emph{SelectionSort}: $\Theta(n^2)$.
		\item \emph{InsertionSort}: $\Omega(n)$, $O(n^2)$.
		\item \emph{ShellSort}: $\Theta(n)$, $O(n^\frac{3}{2})$.
		\item \emph{MergeSort}: $\Theta(n\log n)$.
		\item \emph{HeapSort}: $\Theta(n\log n)$.
		\item \emph{QuickSort}: $\Omega(n\log n)$, $O(n^2)$.
	\end{itemize}
\end{multicols}
Tutti questi algoritmi sono basati su confronti in quanto le decisioni sull'ordinamento vengono prese in base al confronto ($<$, $=$, $>$)tra due valori. Gli algoritmi migliori sono
$O(n\log n)$ e \emph{InsertionSort} e \emph{ShelloSort} sono pi\`u veloci solo in casi speciali. 
\section{Limite inferiore}
\`E possibile dimostrare che qualunque algoritmo di ordinamento basato su confronti ha una complessit\`a $\Omega(n\log n)$. Si consideri un qualunque algoritmo $A$ basato su confronti e
si assuma che tutti i valori siano distinti senza perdere generalit\`a. L'algoritmo $A$ pu\`o essere rappresentato tramite un albero di decisione, un albero binario che rappresenta i 
confronti fra gli elementi. 
\paragraph{Propriet\`a dell'albero di decisione}
Si intende il cammino radice-foglia in un albero di decisione la sequenza di confronti eseguiti dall'algoritmo corrispondente. L'altezza dell'albero di decisione \`e il numero di 
confronti eseguiti dall'algoritmo corrispondente nel caso pessimo. 
\paragraph{Lemma $\mathbf{1}$}
Un albero di decisione per l'ordinamento di $n$ elementi contiene almeno $n!$ foglie. 
\paragraph{Lemma $\mathbf{2}$}
Sia $T$ un albero binario in cui ogni nodo interno ha esattamente $2$ figli e sia $k$ il numero delle sue foglie. L'altezza dell'albero \`e almeno $\log k$, ovvero $\Omega(\log k)$. 
\paragraph{Teorema}
Il numero di confronti necessari per ordinare $n$ elementi nel caso peggiore \`e $\Omega(n\log n)$. 
\section{Ulteriori algoritmi di ordinamento}
\subsection{Spaghetti Sort}
Ha complessit\`a $O(n)$.
\begin{enumerate}
	\item si considerino $n$ spaghetti.
	\item Si taglia lo spaghetto $i$-esimo in modo proporzionale all'$i$-esimo valore da ordinare.
	\item Con la mano si afferrano gli $n$ spaghetti e li si appoggiano verticalmente sul tavolo. 
	\item Si prende il pi\`u lungo lo si misura e si mette il valore corrispondente al valore da ordinare. 
	\item Si ripete $(4)$ fino a quando non hai terminato gli spaghetti. 
\end{enumerate}
\subsection{Counting Sort}
Si assuma che i numeri da ordinare siano compresi in un intervallo $[1\dots k]$. Si costruisce un array $B[1\dots k]$ che conta il numero di volte che un valore compreso in $[1\dots k]$
compare in $A$. Ricolloca i valori cos\`i ottenuti nel vettore da ordinare $A$. Qualunque intervallo di cui si conoscono gli estremi pu\`o essree utilizzato nel Counting Sort. 
\begin{algorithm}[H]
\DontPrintSemicolon
\SetKwComment{comment}{$\%$}{}
\SetKwRepeat{Do}{do}{while}
\SetKw{Int}{int}
\SetKw{Float}{float}
\SetKw{Boolean}{boolean}
\SetKw{New}{new}
\SetKw{True}{true}
\SetKw{False}{false}
\SetKw{Not}{not}
\SetKw{And}{and}
\SetKw{Or}{or}
\SetKw{Down}{down}
\SetKw{To}{to}
\SetKw{New}{new}
\SetKw{Return}{return}
\SetKw{Nil}{nil}
\SetKw{Print}{print}
\SetKw{Void}{void}

\SetKwData{Item}{Item}
\SetKwData{Tree}{Tree}
\SetKwData{PriorityQueue}{PriorityQueue}
\SetKwData{Mfset}{Mfset}
\SetKwData{Graph}{Graph}
\SetKwData{N}{n}
\SetKwData{Space}{ }
\SetKwData{Parent}{parent}
\SetKwData{Rank}{rank}
\SetKwData{Set}{Set}
\SetKwData{List}{List}
\SetKwData{F}{f}
\SetKwData{Left}{left}
\SetKwData{Right}{right}
\SetKwData{Dictionary}{Dictionary}
\SetKwData{Mfset}{Mfset}
\SetKwData{PriorityItem}{PriorityItem}
\SetKwData{Node}{Node}
\SetKwData{Point}{Point}
\SetKwData{Stack}{Stack}

\SetKwFunction{Max}{max}
\SetKwFunction{Min}{min}
\SetKwFunction{Insert}{insert}
\SetKwFunction{ListCos}{List}
\SetKwFunction{Head}{head}
\SetKwFunction{Len}{len}
\SetKwFunction{SetCos}{Set}
\SetKwFunction{TreeCos}{Tree}
\SetKwFunction{MinPriorityQueue}{MinPriorityQueue}
\SetKwFunction{DeleteMin}{deleteMin}
\SetKwFunction{MfsetCos}{Mfset}
\SetKwFunction{Find}{find}
\SetKwFunction{Lookup}{lookup}
\SetKwFunction{Merge}{merge}
\SetKwFunction{IsEmpty}{isEmpty}
\SetKwFunction{Adj}{adj}
\SetKwFunction{Push}{push}
\SetKwFunction{Pop}{pop}
\SetKwFunction{Top}{top}
\SetKwFunction{Decrease}{decrease}
\SetKwFunction{StackCos}{StackCos}


\SetKwProg{Fn}{}{}

\SetKwFunction{CountingSort}{countingSort}
\SetKwFunction{}{}
\SetKwFunction{}{}
\SetKwFunction{}{}
\SetKwFunction{}{}
\SetKwFunction{}{}

\caption{\protect\CountingSort{\protect\Item[] A, \protect\Int n, \protect\Int k}}
\Int[] B = \New \Int$[1\dots k]$\;
\For{\Int i = 1 \To k}{
	B[i] = 0\;
}
\For{\Int j = 1 \To n}{
	B[A[j]] += 1\;
}
j = 1\;
\For{\Int i = 1 \To k}{
	\While{B[i] $>$ 0}{
		A[j] = i\;
		j += 1\;
		B[i] -= 1\;
	}
}
\end{algorithm}

\subsubsection{Complessit\`a}
$O(n+k)$, se $k = O(n)$, allora la complessit\`a di Counting Sort \`e $O(n)$. Si noti come Counting Sort non \`e basato su confronti e che se $k = O(n^3)$ questo \`e il peggior algoritmo
visto fino ad ora. 
\subsection{Pigeonhole Sort}
Una variante del Counting Sort in cui i valori non sono numeri interi ma record associati ad una chiave da ordinare, allora si devono usare liste concatenate invece di contare. 
\subsection{Bucket Sort}
Si ipotizza sull'input in cui i valori sono reali uniformemente distribuiti nell'intervallo $[0, 1)$. Qualunque insieme di valori distribuiti uniformemente pu\`o essere noermalizzato
nell'intervallo $[0, 1)$ in tempo lineare. Si diveide l'intervallo in $n$  di dimensione $\frac{1}{n}$ detti bucket che si distribuiscono gli $n$ numeri nei bucket. Per l'ipotesi di 
uniformit\`a il numero atteso di valori nei bucket \`e $1$. Possono essere ordinati con Insertion Sort. 
\section{Conclusioni}
\subsection{Stabilit\`a}
Un algoritmdo di ordinamento si dice stabile se preserva l'oridne iniziale tra due elementi con la stessa chiave. Si pu\`o rendere stabile qualunque algoritmo usando come chiave di 
ordinamento la coppia chiave, posizione iniziale. 
\subsection{Riassunto}
\subsubsection{Insertion Sort}
$\Omega(n)$, $O(n^2)$, stabile, sul posto, iterativo. \`E adatto per piccoli valori, sequenze quasi ordinate. 
\subsubsection{Merge Sort}
$\Theta(n\log n)$, stabile, richiede $O(n)$ spazio aggiuntivo, ricorsivo (richiede $O(\log n)$ spazio nello stack), buona performance in cache e buona parallelizzazione. 
\subsubsection{Heap Sort}
$\Theta(n\log n)$, non stabile, sul posto, iterativo. Cattiva performance in cache, cattiva parallelizzazione. Preferito in sistemi embedded. 
\subsubsection{Quick Sort}
$O(n\log n)$ in media, $O(n^2)$ nel caso peggiore, non stabile, ricorsivo (richiede $O(\log n)$ spazio nello stack). Buona performance in cache, buona parallelizzaizone, buoni fattori
moltiplicativi. 
\subsubsection{Counting Sort}
$\Theta(n+k)$, richiede $O(k)$ memoria aggiuntiva, iterativo, molto veloce quanro $k= O(n)$.
\subsubsection{Pigeonhole Sort}
$\Theta(n + k)$, stabile, richiede $O(n+k)$ memoria aggiuntiva, iterativo, molto veloce quando $k = O(n)$.
\subsubsection{Bucket Sort}
$O(n)$ nel caso i valori siano distribuiti uniformemente, stabile, richiede $O(n)$ spazio aggiuntivo. 
\subsubsection{Shello Sort}
$O(n\sqrt{n})$, stabile, adatto per piccoli valori, sequenze quasi ordinate. 
\subsubsection{Tim Sort}
Algoritmo ibrido basato su Merge Sort e Insertion Sort. Cerca sequenze consecutive gi\`a ordinate. Ha complessit\`a $\Omega(n)$ per le sequenze gi\`a ordinate o $O(n\log n)$ nel caso
pessimo .
